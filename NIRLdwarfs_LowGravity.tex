\documentclass[12pt,preprint]{aastex}
\usepackage{natbib}
\usepackage{graphicx}
%\usepackage{rotating}
\usepackage{lscape}

\slugcomment{Draft \today}
\shorttitle{short title}
\shortauthors{Cruz et al.}

\begin{document}

\title{
Low Gravity Near-Infrared L Dwarfs
}
\author{
Kelle L. Cruz\altaffilmark{1,2,3,4},
Alejandro N\'{u}\~{n}ez\altaffilmark{1,2,3,4}, 
Emily Rice, 
J. Davy Kirkpatrick\altaffilmark{6},
Adam J. Burgasser\altaffilmark{7}, 
Michael Cushing\altaffilmark{8},
%spectral typing code and probably lots of other stuff
}
\altaffiltext{1}{Department of Physics and Astronomy, Hunter College, City University of New York, New York, NY 10065; \email{kellecruz@gmail.com}}
\altaffiltext{2}{Department of Astrophysics, American Museum of
Natural History, New York, NY 10024}
\altaffiltext{3}{Astronomy Department, California Institute of Technology, Pasadena, CA 91125}
\altaffiltext{3}{Visiting Astronomer at the Infrared Telescope Facility, which is operated by the University of Hawaii under Cooperative Agreement no. NCC 5-538 with the National Aeronautics and Space Administration, Science Mission Directorate, Planetary Astronomy Program.}
\altaffiltext{5}{Space Telescope Science Institute, Baltimore, MD 21218 }
\altaffiltext{6}{Infrared Processing and Analysis Center, California Institute of Technology, Pasadena, CA 91125}
\altaffiltext{7}{Massachusetts Institute of Technology, Kavli
Institute for Astrophysics and Space Research, Cambridge, MA 02139}
\altaffiltext{8}{Department of Astronomy, Toledo}
\altaffiltext{9}{Department of Astronomy, Toledo}
\begin{abstract}
Present near-infrared spectra obtained with SpeX Prism data for \# objects.

Optical spectra for new low-gravity objects.

By averaging numerous spectra we present three-band spectral templates for types L0$\gamma$ and L4$\gamma$. 

establish low gravity diagnostics in the NIR.

\end{abstract}
\keywords{Galaxy: stellar content --- solar neighborhood ---
stars: late-type stars: low-mass, brown dwarfs --- stars:
luminosity function, mass function}

\section{Introduction}
This Paper is in anticipation of the discoveries of new L dwarfs and hot planetary mass objects follow up for characterization of NIR spectra obtained from TSpec, NIRES, SpeX, FIRE.

Indeed, those data have been used to identify gravity sensitive features \citep{McGovern04, Gorlova03}.
\citet{Allers07} used low-res data to distinguish low-gravity.
NIR spectra have been the primary spectroscopic follow-up for objects in young associations since optical spectroscopy is unreasonable \citep{Lucas01,Luhman06,Muench07}. Recently, \citet{Lodieu07} got high res for objects in Upper Sco (but forgot the J band!).



 \cite{Bihain10} - low-res NIR data of Pleiades objects
 
 \cite{Bowler10_HR8799}


Cruz09, Allers07, Kirkpatrick Lithium. 0141.

% The first three papers and Papers~VII
% and~VIII \citep{Paper1, Paper2, Cruz02,Paper7,Paper8} concentrate
% on proper-motion selected K and M dwarfs. Paper IV \citep{Paper4}
% describes the discovery of \object{2MASS J18353790+3259545}, an
% M8.5 dwarf within 6~pc of the Sun (which is included in the
% present analysis). Results from a search for ultracool dwarfs
% lying close to the Galactic Plane are presented in \citet[Paper
% VI]{Paper6}.

\section{Near-Infrared Spectroscopy of Nearby Ultracool Dwarfs}
\subsection{Target Selection}
Our campaign to obtain near-infrared (NIR) spectroscopy was initiated as part of the 20~pc census for nearby ultracool dwarfs using 2MASS. Photometric selection criteria were used to identify candidate nearby ultracool dwarfs in the 2MASS catalog. The selection criteria and sample creation are described extensively in \cite{Cruz03,Cruz07,Reid08} (Papers~III, V, and IX). Candidates were then targeted for far-red optical and/or near-infrared spectroscopic follow-up to obtain spectral types and photometric distances. Our NIR follow-up campaign was focused on candidates nearby ultracool dwarfs that were too faint for optical follow-up. In addition, in an effort to compile an extensive library of NIR spectra for ultracool dwarfs, we also targeted optically-confirmed nearby M and L dwarfs from our own work as well as from the literature. These included field objects as well as benchmark ultracool dwarfs that are confirmed members of star forming regions, young moving groups, and open clusters. L dwarfs were observed with a higher priority over M dwarfs.
                 
\subsection{IRTF SpeX Observations}
\newcommand{\spectra}{193}
\newcommand{\prism}{176} %193-17
\newcommand{\sxd}{17}
\newcommand{\nironly}{20}
\newcommand{\dupes}{16}
\newcommand{\runs}{eight}
\newcommand{\objects}{177} %193-16


We have obtained \spectra~spectra of \objects~late-M and L dwarfs with the SpeX spectrograph \citep{Spex} on the 3~m NASA Infrared Telescope Facility. Observations were obtained over \runs~runs during 2003--2011. The observed objects, their 2MASS near-infrared photometry, optical spectra types, and observation dates are listed in Table~\ref{tab:nir_data}. 

We primarily used the low-resolution prism mode of SpeX to provide a wavelength coverage of 0.8--2.5~$\micron$ in a single order and an average resolution of $\delta\lambda/\lambda\sim120$ with a 0$\farcs$5-wide slit. 

Reduced spectra of the ultracool dwarfs are available for download on the SpeX Prism Spectral Library maintained by Adam Burgasser. Objects that turned out not to be ultracool dwarfs and instead reddened M dwarfs or galaxies are discussed in the Appendix.

\section{Optical Spectra}
obtained new optical spectra.
Following the work of \cite{cruz09_lowg}. 

with LRIS spectrograph on Keck.

Spectral types listed in Table \ref{tab:opt}.

\section{Optical and NIR L Dwarf Data Combined}

We have combined data from the literature with our own extensive spectral library to create a samply of \sample~L dwarfs that have both far-red optical (6000–-9000 \A) and SpeX prism spectra.

We chose to limit the sample to objects with NIR spectra from SpeX prism for two main reasons: 1) SpeX prism mode is a single order and thus the systematic uncertainties introduced by having to stitch spectra together are minimized and 2) SpeX prism mode has proved to be a popular workhorse instrument for brown dwarf spectroscopic follow-up and as a result, there is a plethora of data available. We chose not to include NIR spectra from other instruments in order  to avoid smoothing and resolution-matching issues and, because there is so much SpeX prism data, our analysis would not substantively benefit from more data.

The optical spectra, on the other hand, were obtained with long slit spectrographs on several different instruments (e.g., RC Spectrograph on Mayall and Blanco, GMOS on Gemini-North and South, and LRIS on Keck). However, they all have similar resolution (R$\sim$1000) and were all reduced in essentially the same way. Most data were reduced with the \emph{doslit} routine in the CCDRED IRAF package while the GMOS data were reduced with the Gemini GMOS IRAF package. One inconsistency in our library of optical spectra is telluric correction: most Keck-LRIS data are corrected, while most others are not. However this does not affect the spectral type.


The spectra, grouped by \emph{optical} spectral type, are shown in Figures~\ref{fig:L0young}--\ref{fig:L4young}. Unlike previous studies, we plot and analyze the three bands (J, H, and K) separately. This effectively de-reddens the spectra and allows the spectral similarities to be seen more clearly. Each band is normalized across its entire wavelength range. Normal and low gravity objects are plotted separately to further isolate objects with similar underlying physical properties.

\subsection{L Dwarf NIR Spectral Averages Templates}
Given our relatively large dataset of L dwarfs, we have combined the objects of the same optical spectral type to make spectral average templates. In choosing objects to include in the spectral averages of field objects, only "normal" objects were included and peculiar, suspected blue, and suspected dusty or young objects were excluded. The data used are shown in  Figures~~\ref{fig:L0field}--\ref{fig:L8field} along with the resultant template (black). The number of objects included in each template ranges from 3--10. The spectra were normalized in each of the bands (JHK) and then median combined. 

\subsection{The Near-Infrared Low-Gravity L Dwarf Spectral Sequence}
The templates are shown overplotted on top of each other in Figure \ref{fig:spec_sequence}. 



\subsection{Spectral Features of Suspected Low-Gravity L Dwarfs Compared to Field L Dwarfs}

\subsubsection{L0$\gamma$}
In Figure~\ref{fig:L0g-field} the L0$\gamma$ spectral average template (black) is compared to all the other field L dwarf templates (color). (As shown in Figure~\ref{fig:L0young}, seven optically classified objects were combined to make the L0$\gamma$ template and all have remarkably similar spectra.) The J and H bands of the L0$\gamma$ template are distinctive from those of all other L dwarfs. In particular -- as was first described in 2M~0141-4633, the L0$\gamma$ prototype, by \cite{Kirkpatrick06} -- the J band of the L0$\gamma$ type displays weak FeH absorption at 0.99--1.007 $\mu$m and strong VO absorption at 1.05--1.08 $\mu$m and triangular-shaped H band compared to the field objects. Aside from these notable differences, the J-band shape is most similar to L0 and L1 field objects, but with a slightly redder slope. The L0$\gamma$ K-band shape, on the other hand, is only subtly different from the L0 and L1 field objects, with a slightly flatter and less round shape and weaker CO absorption in the 2.3~$\mu$m. 
\emph{We assert that in order to credibly distinguish an L0$\gamma$ from a field object in the NIR, a K-band spectrum is insufficient and H- and/or J-band data is required.}

\subsubsection{L2$\gamma$}
In Figure~\ref{fig:L2g-field} the L2$\gamma$ spectral average template (black) is compared to all the other field L dwarf templates (color). (As shown in Figure~\ref{fig:L2young}, two optically classified objects were combined to make the L2$\gamma$ template and both have remarkably similar spectra.) 
The L2$\gamma$ type is distinctive from field dwarfs in all three J, H, and K-bands. 
Even more pronounced than in the L0$\gamma$, the L2$\gamma$ J-band spectrum has weaker FeH and stronger VO compared to field L dwarfs. 
In addition, there is also a hint of weaker \ion{K}{1} absorption at 1.25~$\mu$m. Aside from these differences in particular absorption features, the overall shape of the L2$\gamma$ J-band is most similar to the L5 field template.
The L2$\gamma$ H-band most closely resembles that of the L7 field template but with the hallmark  triangular-shape. 
Finally, the L2$\gamma$ K-band shape resembles that of early-type (L0--L2) field objects, except with a redder overall slope and weaker CO at 2.3~$\mu$m. 
\emph{We assert that in order to credibly distinguish an L2$\gamma$ from a field object in the NIR, either a J, H, or K-band spectrum is sufficient.}

\subsubsection{L4$\gamma$}
In Figure~\ref{fig:L4g-field}, the L4$\gamma$ spectral average template (black) is compared to all the other field L dwarf templates (color). (As shown in Figure~\ref{fig:L2young}, four optically classified objects were combined to make the L4$\gamma$ template.) 
The NIR spectra of the L4$\gamma$ is not as distinctive from the field objects as the earlier type $\gamma$ objects. 
The J band of the L4$\gamma$ template looks very similar to the field L4 but with weaker 1$\micron$ FeH absorption and a redder overall slope. 
Just like the earlier-type $\gamma$ templates, the triangular-shaped H band is quite distinctive from the H band shape of the field objects. And similar to the L2$\gamma$, the H-band of the L4$\gamma$ template most resembles that of the L7 field template.
Finally, the L4$\gamma$ K-band shape most resembles that of the earlier-type (L1--L3) field templates, but with weaker CO absorption at 2.3~$\mu$m.
\emph{We assert that in order to credibly distinguish an L4$\gamma$ from a field object in the NIR, a K-band spectrum is insufficient and H- and/or J-band data is required.}

\subsection{Notes on Individual Objects}
% \subsubsection{2M 1821+1414}
% \cite{Looper08_dusty} identify as dusty object with silicate feature.
% Looks like normal in both optical and NIR. Optical = L4.5, while NIR falls within the typical range of spectral features exhibited by other normal field objects. that is, consistent with typical features seen in L3, L4, and L5 templates. We see no evidence of peculiarity or "dustiness". 
% 
% \subsubsection{2M 2317-4838}
% \cite{Reid08} L4 in the optical but \cite{Kirkpatrick10} call it L6.5pec in the NIR and red optical. However, our band-by-band analysis shows the NIR spectrum to be broadly consistent with other L4s. Albeit with higher J band (similar to 0512) and H band (similar to 1821+1414). Consider calling it L4pec.  

\subsubsection{2148+4003 2244+2043}
both pec. very red L6s. and Gizis' very red WISE object.
                                     
\section{Conclusions/Summary}
progress towards understanding/decoding the underlying physics of L dwarf atmospheres and, in turn, planetary atmospheres.

\acknowledgments
We would like to acknowledge the IRTF telescope operators
and support staff at 
This research was partially supported by a grant from the NASA/NSF
NStars initiative, administered by JPL, Pasadena, CA.  This publication makes use of data products from the Two Micron All Sky Survey, which is a joint project of the University of Massachusetts and Infrared Processing and Analysis
Center/California Institute of Technology, funded by the National Aeronautics and Space Administration and the National Science Foundation; the NASA/IPAC Infrared Science Archive, which is operated by the Jet Propulsion Laboratory/California Institute of Technology, under contract with the National Aeronautics and Space Administration.  This research has made use of the SIMBAD database, operated at CDS, Strasbourg, France.
Mauna Kea is a special place.

Facilities: 
\facility{FLWO:2MASS}, 
\facility{CTIO:2MASS},
\facility{Mayall (MARS)}, 
\facility{Blanco (RC Spec)},
\facility{Gemini:South (GMOS)}, 
\facility{Gemini:Gillett (GMOS),
\facility{KPNO:2.1m (GoldCam)}, 
\facility{CTIO:1.5m (RC Spec)},
\facility{ARC (DIS II)}}
                                     
\bibliographystyle{apj}
\bibliography{bib_all}
\clearpage 


\section{figures}

\begin{landscape}

\begin{figure}
	\epsscale{0.80}
	\plotone{figures/L0G.pdf}
	\caption{Normalized optical and near-infrared low resolution spectra of field L0 type
dwarfs. The spectra are normalized individually by band (optical, J, H, and K bands)
using the entire band range. The most prominent atomic and molecular features are indicated
(do we need to clarify what H2 CIA is?). Superimposed in black in the near-infrared
bands is the L0 field template spectrum calculated using a weighted average of all the
field objects plotted. Field objects with the same effective temperature —as deducted by
the tight dispersion in the optical band— exhibit a wider dispersion in the
near-infrared bands, indicating different effects of dust and clouds on their
atmospheres.}
\label{fig:L0young}
\end{figure}
\clearpage


\begin{figure}
	\plotone{figures/L1G.pdf}
	\caption{}
	\label{fig:L1young}
\end{figure}


\begin{figure}
	\plotone{figures/L2G.pdf}
	\caption{}
	\label{fig:L2young}
\end{figure}

\begin{figure}
	\plotone{figures/L3G.pdf}
	\caption{}
	\label{fig:L3young}
\end{figure}

\begin{figure}
	\plotone{figures/L4G.pdf}
	\caption{}
	\label{fig:L4young}
\end{figure}

% \begin{figure}
% 	\plotone{figures/L5G.pdf}
% 	\caption{}
% 	\label{fig:L5young}
% \end{figure}

\end{landscape}

\begin{figure}
	\plotone{figures/L0g-field.pdf}
	\caption{The L0$\gamma$ spectral average template (black) compared to the field L dwarf templates (color). The field templates that most resemble the L0$\gamma$ are plotted with thicker lines. We assert that in order to use NIR spectra to credibly distinguish an L0$\gamma$ from a field object in the NIR, a K band spectrum is insufficient and H and/or J band data is required.}
	\label{fig:L0g-field}
\end{figure}

\begin{figure}
	\plotone{figures/L2g-field.pdf}
	\caption{The L2$\gamma$ spectral average template (black) to the field L dwarf templates (color). The field templates that most resemble the L2$\gamma$ are plotted with thicker lines. We assert that in order to use NIR spectra to credibly distinguish an L2$\gamma$ from a field object in the NIR, either a J, H, or K-band spectrum is sufficient.}
	\label{fig:L2g-field}
\end{figure}

\begin{figure}
	\plotone{figures/L4g-field.pdf}
	\caption{The L4$\gamma$ spectral average template (black) to the field L dwarf templates (color). The field templates that most resemble the L4$\gamma$ are plotted with thicker lines. We assert that in order to use NIR spectra to credibly distinguish an L4$\gamma$ from a field object, a K-band spectrum is insufficient and H and/or J band data is required.}
	\label{fig:L4g-field}
\end{figure}


\clearpage

\section{tables}
%TABLES
% \input{tables/tab_optnir.tex}
% \clearpage
% 
% % new nir data
\input{tables/tab_nir_lowg.tex}
\clearpage
% 
% %previously published data used for opt-nir analysis
% \input{tables/tab_addtl_data.tex}
% \clearpage


\end{document}
