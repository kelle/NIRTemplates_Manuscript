\documentclass[modern,trackchanges]{aastex61}

\usepackage{natbib}

\received{April 22, 2016}
\revised{October 30, 2017}
\accepted{}

% \slugcomment{Resubmitted April 18, 2017, Draft \today}
\shorttitle{Optically-Anchored Analysis of L Dwarfs}
\shortauthors{Cruz et al.}

\newcommand{\sample}{171}

%field + low g should equal sample
\newcommand{\optField}{143}
\newcommand{\optLowG}{27}

%beta + gamma should equal lowg
\newcommand{\optBeta}{6}
\newcommand{\optGamma}{21}

\newcommand{\NewOptSpectra}{18} %New optical Data
\newcommand{\NewOptObjects}{16} %objects with new optical data

%NIR
\newcommand{\NewPrismSpectra}{155} %New Prism Data
\newcommand{\NewPrismObjects}{147} % New Prism Breakdown

\begin{document}

\title{
Meeting the Cool Neighbors XII: An Optically-Anchored Analysis of the Near-Infrared Spectra of L Dwarfs.
}

\correspondingauthor{Kelle L. Cruz}
\email{kellecruz@gmail.com}

\author[0000-0002-1821-0650]{Kelle L. Cruz}
\altaffiliation{Visiting Astronomer at the Infrared Telescope Facility, which is operated by the University of Hawaii under Cooperative Agreement no. NCC 5-538 with the National Aeronautics and Space Administration, Science Mission Directorate, Planetary Astronomy Program.}
\affil{Department of Physics and Astronomy, Hunter College, City University of New York, 695 Park Avenue, New York, NY 10065}
\affiliation{Physics, Graduate Center of the City University of New York, 365 5th Avenue, New York, NY, 10016 USA}
\affiliation{Department of Astrophysics, American Museum of Natural History, Central Park West at 79th St, New York, NY, 10024, USA}

\author[0000-0002-8047-1982]{Alejandro N\'{u}\~{n}ez}
\altaffiliation{Visiting Astronomer at the Infrared Telescope Facility, which is operated by the University of Hawaii under Cooperative Agreement no. NCC 5-538 with the National Aeronautics and Space Administration, Science Mission Directorate, Planetary Astronomy Program.}
\affil{Department of Astronomy, Columbia University, 550 West 120th Street, New York, NY 10027, USA}
\affiliation{Department of Physics and Astronomy, Hunter College, City University of New York, 695 Park Avenue, New York, NY 10065, USA}


\author[0000-0002-6523-9536]{Adam J. Burgasser} %SpeX Data & template code
\affil{Center for Astrophysics and Space Sciences, University of California, San Diego, San Diego, CASS, M/C 0424 9500 Gilman Drive, La Jolla, CA 92093-0424, USA}

\author[0000-0002-9879-1183]{Ellianna Abrahams} %nir spectral types comparsion
\affil{Department of Astrophysics, American Museum of Natural History, Central Park West at 79th St, New York, NY, 10024, USA}
\affiliation{Department of Physics, City College of New York, City University of New York, 160 Convent Ave, New York, NY 10031, USA}

\author[0000-0002-3252-5886]{Emily L. Rice} %introduction and conclusions
\affil{Department of Engineering Science and Physics, College of Staten Island, City University of New York, Staten Island, NY 10310, USA}
\affiliation{Physics, Graduate Center of the City University of New York, 365 5th Avenue, New York, NY, 10016 USA}
\affiliation{Department of Astrophysics, American Museum of Natural History, Central Park West at 79th St, New York, NY, 10024, USA}

\author[0000-0003-0531-8547]{I. Neill Reid} %IRTF observations
\altaffiliation{Visiting Astronomer at the Infrared Telescope Facility, which is operated by the University of Hawaii under Cooperative Agreement no. NCC 5-538 with the National Aeronautics and Space Administration, Science Mission Directorate, Planetary Astronomy Program.}

\affil{Space Telescope Science Institute, 3700 San Marin Drive, Baltimore, MD 21218, USA}
\author{Dagny Looper}
%\affil{NYU Tisch School of the Arts, 721 Broadway, 10th floor, New York, NY 10003}


\begin{abstract}
Discrepancies between competing optical and near-infrared (NIR) spectral typing systems for L dwarfs have motivated us to search for a classification scheme which ties the optical and NIR schemes \added{together,} and addresses complexities in the spectral morphology.
We use new and extant optical and NIR spectra to compile a sample of~\sample~L dwarfs, including \optLowG~low-gravity $\beta$ and $\gamma$ objects, with spectral coverage from 0.6--2.4$\micron$.
We present \NewPrismSpectra~new low-resolution NIR spectra and ~\NewOptSpectra~new optical spectra.
We utilize a method for analyzing NIR spectra that partially removes the broad band spectral slope and reveals similarities in the absorption features between objects of the same optical spectral type.
Using the optical spectra as an anchor, we generate near-infrared spectral average templates for L0–-L8, L0–-L4$\gamma$, and L0--L1$\beta$ type dwarfs.
These templates reveal that NIR spectral morphologies are correlated with the optical types.
\added{They also} show the range of spectral morphologies spanned by each spectral type.
We compare low-gravity and field-gravity templates to provide recommendations on the minimum required observations for credibly classifying low-gravity spectra using low-resolution NIR data.
We use the templates to evaluate the existing NIR spectral standards and propose new ones where appropriate.
Finally, we build on the work of \cite{Kirkpatrick10} to provide a spectral typing method \replaced{which}{that} is tied to the optical and can be used when only \emph{H}- or \emph{K}-band data are available.
The methods we present here provide resolutions to several long standing issues with classifying L dwarf spectra and
could also be the foundation for a spectral classification scheme for cloudy exoplanets.
\end{abstract}
\keywords{brown dwarfs --- Galaxy: stellar content --- solar neighborhood --- stars: low-mass}

\section{Introduction}
% What are L dwarfs
The L~dwarf spectral class is comprised of both the lowest mass stars and brown dwarfs which are cooler than M~dwarfs ($\lesssim$2200~K) and \replaced{hotter}{warmer} than T~dwarfs ($\gtrsim$1500~K).
The spectral energy distribution of L dwarfs deviates significantly from a blackbody function, and their spectral morphology is dominated by pressure-broadened alkali lines and deep molecular absorption bands.
Building upon the initial discovery and classification of these objects \citep{Becklin88_GD165B,K99,Martin99,Kirkpatrick05}, previous papers in this series focused on compiling a volume-limited sample of field late-M and L dwarfs in order to measure their number density and luminosity function \citep[Papers V, IX, and X]{Cruz03,Cruz07,Reid08}.
In this final paper of the series, we focus on the analysis of the NIR spectra of L dwarfs obtained while building that volume-limited sample.

% L dwarf classification in the OPTICAL
\added{Spectral classification aims to provide a method of succinctly and accurately inferring the physical properties of a celestial object based on its spectral morphology.
Achieving that aim requires that spectral analyses of the same object at different wavelengths give self-consistent results.}
The first classification scheme developed for L dwarfs \replaced{is}{was} based on their spectral morphology at red-optical wavelengths using spectra typically covering 6000--9000~\AA~at $R\sim1000$ \citep{Martin99,K99,Kirkpatrick05}.
This classification scheme \replaced{relies}{relied} on the prominent spectral absorption features due to \ion{K}{1} broadening, metal oxides (VO, TiO), metal hydrides (CrH, FeH), and alkali metals (Cs, Rb).
Red-optical spectra of L dwarfs are on the Wien side of the underlying blackbody spectrum, and the spectral morphology in that wavelength regime is dominated by temperature effects, with surface gravity and metallicity making secondary contributions.
The effects on the optical spectra of L dwarfs due to low gravity have been described by \citet{Martin99,Kirkpatrick06,Kirkpatrick08, Cruz09_lowg,Luhman09_taurus}.
Building on the \citet{K99} scheme, \citet{Cruz09_lowg} also defined a classification scheme for low-gravity optical spectral types using $\beta$, $\gamma$, and $\delta$ suffixes to indicate intermediate-gravity, very low-gravity and extremely low-gravity spectral features, respectively.
\added{There are also L subdwarfs and so-called ``blue'' L dwarfs with low metalliticy and old ages.
Since there are so few known,} they are typically classified using the field-gravity standards \citep{Gizis:2006jl,Burgasser07_subdwarfs, Burgasser08_blue, Kirkpatrick10, Kirkpatrick:2014kv}.

% NEED FOR NIR CLASSIFICATION
While the classification and analysis techniques in the optical are valuable, reliable diagnostics based on near-infrared (NIR) data are necessary for two primary reasons:
First, with intrinsically cool surface temperatures (1500--2200~K) and low luminosities ($-5~\lesssim~log(L_*/L_\sun)~\lesssim~-3.5$), L dwarfs are relatively faint in the optical and emit most of their flux at near-infrared wavelengths \citep[e.g.,][]{Filippazzo:2015dv}.
As a result, it is more efficient to obtain NIR spectra with higher signal-to-noise ratios than at optical wavelengths.
Second, the workhorse spectrographs of the next decade are optimized for the NIR \replaced{. For example}{(e.g.}, the Folded-port Infrared Echellette (FIRE) on Magellan \citep{Simcoe:2013kh}, the Near-Infrared Echelle Spectrograph (NIRES) on Keck, and the Near Infrared Spectrograph (NIRSpec) on the James Webb Space Telescope \citep{Ferruit:2012em}\added{)}.

% Trouble with L dwarf spectral classification in the NIR
Ever since their discovery, the observational properties of L dwarfs in the NIR have proved more complex to interpret than their behavior at optical wavelengths.
Spectral classification and inference of unambiguous physical properties based on near-infrared data has been fraught with inconsistencies and degeneracies.
This was first recognized in their photometric properties where, unlike M~dwarfs, L~dwarfs of the same optically-based spectral type span a large range of $J-K$ colors \citep[e.g.,][]{Leggett:2003tm}.
There have been numerous efforts to classify the NIR spectra of L dwarfs \citep{Reid01_NIR, Testi01, Geballe02,Burgasser07_subdwarfs}, two of the most widely adopted have been those of \citet{Kirkpatrick10} and \citet{Allers:2013hk}, hereafter K10 and AL13, respectively.
K10 developed a classification sequence based on spectral standards which hinges on the \emph{zJ}-band spectral morphology and uses the \emph{H} and \emph{K} bands to assign `red' or `blue' suffixes based on the overall spectral slope.
The system of AL13 defined quantitative spectral indices to identify spectral type and surface gravity class.
While both of these systems result in types broadly consistent with the optically-determined types, neither use the entire NIR spectral region available from the ground \citep{Kirkpatrick05}.

The relative difficulty of NIR spectral analysis compared to the optical can be attributed to two primary causes:
First, the NIR spectra of L dwarfs are inherently more sensitive to underlying physical properties.
While temperature dominates the spectral variations in the optical, gravity, metallicity, and condensate cloud differences play a significant role on the observed NIR spectral features \citep{Jones:1997hb,Knapp04}.
Thus, different underlying masses, compositions, and cloud properties cause a wider range of spectral morphologies at a given temperature (spectral type) resulting in the range of NIR colors.
This sensitivity to second order parameters has made a satisfying NIR-based classification scheme for L dwarfs elusive.
Second, much of the NIR data for L dwarfs has been collected with the SpeX spectrograph in prism mode, covering 0.8--2.5~$\micron$ at $R\sim200$, a proportionately larger wavelength regime and lower resolution than the optical.
This larger wavelength coverage provides more ``room'' for the diversity of spectral features to be apparent.

The discord between the classification of L dwarfs at different wavelengths and resolutions is reminiscent of the parable of a group of investigators who are asked to describe an elephant based only on a small part of the animal's anatomy.
Each one describes an entirely different part (tusk, trunk, ears, legs) and while none of them are wrong, they are unable to form a coherent description of the elephant.
In the case of the analysis of L dwarf spectra, the different parts of the spectrum are sensitive to different underlying physical properties just like the length of an elephant's tusk, which keeps growing throughout adulthood, is more indicative of its age than the length of its trunk, which stops growing once the animal matures.
Further, due to the differences in wavelength coverage, resolution, \added{and optical depth probed}, the optical spectra are similar to a close up of the elephant's side while the SpeX Prism spectra are like a zoomed out view including the trunk, ears, and tusks providing a more complete view of a complex animal.

Given how important the NIR spectra are as a diagnostic for the fundamental properties of both L~dwarfs and giant exoplanets, we have undertaken a concerted effort to disentangle and isolate the spectral features due to temperature and gravity.

%TABLE OF CONTENTS
In this paper, we present results from a multi-wavelength analysis which include an empirical, descriptive near-infrared spectral classification system for field- and low-gravity L~dwarfs that is anchored to the optical classification schemes.
The sample of L dwarfs with both optical and near-infrared spectra which our analysis is based on is described in \S~ \ref{sec:sample}.
In \S~\ref{sec:templates} we describe the creation of spectral templates for field L0--L8, low-gravity L0$\gamma$--L4$\gamma$, and intermediate-gravity L0$\beta$ \& L1$\beta$ spectral types, including a method of normalizing in individual bands in order to remove color information and focus on characterization of morphology in the \emph{zJ}, \emph{H}, and \emph{K} bands individually.
We analyze and discuss the field templates in \S\S~\ref{sec:fieldg}.
In \S~\ref{sec:lowg}, we discuss the low-gravity templates and identify methods for distinguishing low-gravity objects from field-gravity objects.
In \S~\ref{sec:classification} we detail a procedure for determining the spectral type of an object from the comparison of observed low-resolution \emph{zJ}-, \emph{H}-, and/or \emph{K}-band spectra to our templates.
Finally, in \S~\ref{sec:summary} we summarize our conclusions.

\section{A Sample of L dwarfs with Optical and NIR Spectra}
\label{sec:sample}

Our campaign to obtain spectroscopy of L dwarfs was initiated as part of our effort to complete the 20~pc census of nearby ultracool dwarfs using 2MASS.
Photometric selection criteria were used to identify candidate nearby ultracool dwarfs in the 2MASS catalog.
The selection criteria and sample creation are described extensively in Papers~V,~IX, and~X.
Candidate objects were targeted for far-red optical and/or near-infrared spectroscopic follow-up to obtain spectral types and photometric distances.
During these observing runs, we also targeted confirmed nearby M and L dwarfs from the literature that lacked optical and/or NIR spectra.

As described in more detail below, the far-red optical observations were obtained with several different instruments but we restrict the analysis in this paper to objects with NIR data obtained with the SpeX spectrograph \citep{Spex} on the 3~m NASA Infrared Telescope Facility using the low-resolution prism mode.
We have combined \replaced{these}{new SpeX prism} data, new optical data, and extant spectra from the literature to create a sample of \sample~L~dwarfs with both far-red optical (6000--9000~\AA) and low-resolution SpeX prism spectra (0.85--2.4~$\micron$) which we refer to collectively as the ``optical-NIR sample''.

All of these data are available as Data behind the Figures and from the BDNYC Database\footnote{The BDNYC Database: \url{http://database.bdnyc.org/}}. \added{The website interface to the database relies on AtrodbWeb \citep{Rodriguez:2016ea}. Version 1 of the database \citep{Filippazzo:2016go} includes data from \citet{Filippazzo:2015dv} while version 2 includes all the data used in this paper.}

\subsection{Far-Red Optical Spectra}
\label{sec:obs_new_opt}

%describe optical spectra
For an L dwarf to be included in the optical-NIR sample, it was required to have an optical spectrum of high enough signal-to-noise ratio ($\gtrsim$50) to reliably determine a spectral type.
The optical spectra were obtained with long-slit spectrograph-telescope combinations all with similar resolving power~($\lambda/\Delta\lambda\sim1000$).
The instrumental setup and data reduction techniques used for each spectrum are described in the references listed in Tables~\ref{tab:field_template}--\ref{tab:lowg_excluded}.
Every spectrum was visually compared to spectral standards in order to estimate spectral types using the schemes of \citet{K99} and \cite{Cruz09_lowg} for field- and low-gravity spectra respectively.

Included in the sample are new optical spectra for 16 objects from the Low Resolution Imaging Spectrometer on the 10~m Keck Telescope \citep[LRIS]{LRIS} and one from the Low Dispersion Survey Spectrograph on the Magellan 6.5~m Clay Telescope \citep[LDSS-3]{LDSS2}.
The targets, telescopes, and observation dates are listed in Table~\ref{tab:newopt}.
The data were obtained and reduced in the same manner as described in \citet{Kirkpatrick10}.
In Figure~\ref{fig:spthist}, these new far-red optical data are indicated in the spectral type distribution of the entire optical-NIR sample with left hatching.
The spectra are shown in Figure Set~\ref{fig:newopt}; the FITS spectra can be found as a supplementary tar.gz archive and from the BDNYC Database.

\figsetstart
\figsetnum{1}
\figsettitle{New Far-Red Optical Spectra}

\figsetgrpstart
\figsetgrpnum{1.1}
\figsetgrptitle{Spectrum of 0032-4405}
\figsetplot{0032-4405_1093.pdf}
\figsetgrpnote{New far-red optical spectra listed in Table~\ref{tab:newopt}.
Unlike many spectra of L dwarfs, these data have been telluric corrected.
The FITS spectra can be found as a supplementary tar.giz archive and from the BDNYC Database.}
\figsetgrpend

\figsetgrpstart
\figsetgrpnum{1.2}
\figsetgrptitle{Spectrum of 0055+0134}
\figsetplot{0055+0134_1103.pdf}
\figsetgrpnote{New far-red optical spectra listed in Table~\ref{tab:newopt}.
Unlike many spectra of L dwarfs, these data have been telluric corrected.
The FITS spectra can be found as a supplementary tar.giz archive and from the BDNYC Database.}
\figsetgrpend

\figsetgrpstart
\figsetgrpnum{1.3}
\figsetgrptitle{Spectrum of 0117-3403}
\figsetplot{0117-3403_470.pdf}
\figsetgrpnote{New far-red optical spectra listed in Table~\ref{tab:newopt}.
Unlike many spectra of L dwarfs, these data have been telluric corrected.
The FITS spectra can be found as a supplementary tar.giz archive and from the BDNYC Database.}
\figsetgrpend

\figsetgrpstart
\figsetgrpnum{1.4}
\figsetgrptitle{Spectrum of 0210-3015}
\figsetplot{0210-3015_1125.pdf}
\figsetgrpnote{New far-red optical spectra listed in Table~\ref{tab:newopt}.
Unlike many spectra of L dwarfs, these data have been telluric corrected.
The FITS spectra can be found as a supplementary tar.giz archive and from the BDNYC Database.}
\figsetgrpend

\figsetgrpstart
\figsetgrpnum{1.5}
\figsetgrptitle{Spectrum of 0241-0326}
\figsetplot{0241-0326_496.pdf}
\figsetgrpnote{New far-red optical spectra listed in Table~\ref{tab:newopt}.
Unlike many spectra of L dwarfs, these data have been telluric corrected.
The FITS spectra can be found as a supplementary tar.giz archive and from the BDNYC Database.}
\figsetgrpend

\figsetgrpstart
\figsetgrpnum{1.6}
\figsetgrptitle{Spectrum of 0518-2756}
\figsetplot{0518-2756_578.pdf}
\figsetgrpnote{New far-red optical spectra listed in Table~\ref{tab:newopt}.
Unlike many spectra of L dwarfs, these data have been telluric corrected.
The FITS spectra can be found as a supplementary tar.giz archive and from the BDNYC Database.}
\figsetgrpend

\figsetgrpstart
\figsetgrpnum{1.7}
\figsetgrptitle{Spectrum of 0536-1920}
\figsetplot{0536-1920_586.pdf}
\figsetgrpnote{New far-red optical spectra listed in Table~\ref{tab:newopt}.
Unlike many spectra of L dwarfs, these data have been telluric corrected.
The FITS spectra can be found as a supplementary tar.giz archive and from the BDNYC Database.}
\figsetgrpend

\figsetgrpstart
\figsetgrpnum{1.8}
\figsetgrptitle{Spectrum of 1538-1953}
\figsetplot{1538-1953_400.pdf}
\figsetgrpnote{New far-red optical spectra listed in Table~\ref{tab:newopt}.
Unlike many spectra of L dwarfs, these data have been telluric corrected.
The FITS spectra can be found as a supplementary tar.giz archive and from the BDNYC Database.}
\figsetgrpend

\figsetgrpstart
\figsetgrpnum{1.9}
\figsetgrptitle{Spectrum of 1551+0941 from 2009}
\figsetplot{1551+0941_1353.pdf}
\figsetgrpnote{New far-red optical spectra listed in Table~\ref{tab:newopt}.
Unlike many spectra of L dwarfs, these data have been telluric corrected.
The FITS spectra can be found as a supplementary tar.giz archive and from the BDNYC Database.}
\figsetgrpend

\figsetgrpstart
\figsetgrpnum{1.10}
\figsetgrptitle{Spectrum of 1551+0941 from 2010}
\figsetplot{1551+0941_401.pdf}
\figsetgrpnote{New far-red optical spectra listed in Table~\ref{tab:newopt}.
Unlike many spectra of L dwarfs, these data have been telluric corrected.
The FITS spectra can be found as a supplementary tar.giz archive and from the BDNYC Database.}
\figsetgrpend

\figsetgrpstart
\figsetgrpnum{1.11}
\figsetgrptitle{Spectrum of 1615+4953 from 2009}
\figsetplot{1615+4953_874.pdf}
\figsetgrpnote{New far-red optical spectra listed in Table~\ref{tab:newopt}.
Unlike many spectra of L dwarfs, these data have been telluric corrected.
The FITS spectra can be found as a supplementary tar.giz archive and from the BDNYC Database.}
\figsetgrpend

\figsetgrpstart
\figsetgrpnum{1.12}
\figsetgrptitle{Spectrum of 1615+4953 from 2010}
\figsetplot{1615+4953_403.pdf}
\figsetgrpnote{New far-red optical spectra listed in Table~\ref{tab:newopt}.
Unlike many spectra of L dwarfs, these data have been telluric corrected.
The FITS spectra can be found as a supplementary tar.giz archive and from the BDNYC Database.}
\figsetgrpend

\figsetgrpstart
\figsetgrpnum{1.13}
\figsetgrptitle{Spectrum of 1711+2326}
\figsetplot{1711+2326_404.pdf}
\figsetgrpnote{New far-red optical spectra listed in Table~\ref{tab:newopt}.
Unlike many spectra of L dwarfs, these data have been telluric corrected.
The FITS spectra can be found as a supplementary tar.giz archive and from the BDNYC Database.}
\figsetgrpend

\figsetgrpstart
\figsetgrpnum{1.14}
\figsetgrptitle{Spectrum of 1726+1538}
\figsetplot{1726+1538_405.pdf}
\figsetgrpnote{New far-red optical spectra listed in Table~\ref{tab:newopt}.
Unlike many spectra of L dwarfs, these data have been telluric corrected.
The FITS spectra can be found as a supplementary tar.giz archive and from the BDNYC Database.}
\figsetgrpend

\figsetgrpstart
\figsetgrpnum{1.15}
\figsetgrptitle{Spectrum of 2034+0827}
\figsetplot{2034+0827_420.pdf}
\figsetgrpnote{New far-red optical spectra listed in Table~\ref{tab:newopt}.
Unlike many spectra of L dwarfs, these data have been telluric corrected.
The FITS spectra can be found as a supplementary tar.giz archive and from the BDNYC Database.}
\figsetgrpend

\figsetgrpstart
\figsetgrpnum{1.16}
\figsetgrptitle{Spectrum of 2137+0808}
\figsetplot{2137+0808_408.pdf}
\figsetgrpnote{New far-red optical spectra listed in Table~\ref{tab:newopt}.
Unlike many spectra of L dwarfs, these data have been telluric corrected.
The FITS spectra can be found as a supplementary tar.giz archive and from the BDNYC Database.}
\figsetgrpend

\figsetgrpstart
\figsetgrpnum{1.17}
\figsetgrptitle{Spectrum of 2249+0044}
\figsetplot{2249+0044_1521.pdf}
\figsetgrpnote{New far-red optical spectra listed in Table~\ref{tab:newopt}.
Unlike many spectra of L dwarfs, these data have been telluric corrected.
The FITS spectra can be found as a supplementary tar.giz archive and from the BDNYC Database.}
\figsetgrpend

\figsetgrpstart
\figsetgrpnum{1.18}
\figsetgrptitle{Spectrum of 2315+0617}
\figsetplot{2315+0617_1503.pdf}
\figsetgrpnote{New far-red optical spectra listed in Table~\ref{tab:newopt}.
Unlike many spectra of L dwarfs, these data have been telluric corrected.
The FITS spectra can be found as a supplementary tar.giz archive and from the BDNYC Database.}
\figsetgrpend

\figsetend

\begin{figure}[b]
  \epsscale{0.5}
    \plotone{0032-4405_1093.pdf}
    \plotone{0055+0134_1103.pdf}
    \plotone{0117-3403_470.pdf}
    \plotone{0210-3015_1125.pdf}
    \plotone{0241-0326_496.pdf}
    \plotone{0518-2756_578.pdf}
    \plotone{0536-1920_586.pdf}
    \plotone{1538-1953_400.pdf}
  \caption{New far-red optical spectra listed in Table~\ref{tab:newopt}. Unlike many spectra of L dwarfs, these data have been telluric corrected. The FITS spectra can be found as a supplementary tar.giz archive and from the BDNYC Database. Additional images are available in the Figure Set in the electronic edition of the journal.\label{fig:newopt}}
\end{figure}

\begin{figure}
  \figurenum{1.2}
  \epsscale{0.5}
    \plotone{1551+0941_1353.pdf}
    \plotone{1551+0941_401.pdf}
    \plotone{1615+4953_403.pdf}
    \plotone{1615+4953_874.pdf}
    \plotone{1711+2326_404.pdf}
   \plotone{1726+1538_405.pdf}
    \plotone{2034+0827_420.pdf}
    \plotone{2137+0808_408.pdf}
    \caption{New far-red optical spectra continued. Will be electronic only.}
\end{figure}

\begin{figure}
  \figurenum{1.3}
  \epsscale{0.5}
    \plotone{2249+0044_1521.pdf}
    \plotone{2315+0617_1503.pdf}
    \caption{New far-red optical spectra continued. Will be electronic only.}
\end{figure}
\clearpage

% new optical data
%!TEX root = /Users/kelle/Dropbox/Pubs IN PROGRESS/Paper XII NIR DATA/Cruz-Nunez/NIRLdwarfs_Normal.tex
\begin{deluxetable}{lcclcllll}
%\tablewidth{0pt}
%\tabletypesize{\footnotesize}
%\tabletypesize{\small}
\tabletypesize{\scriptsize}
%\tabletypesize{\tiny}
%\rotate
\tablewidth{0pt}
\tablecolumns{9}
\tablecaption{New Far-Red Optical Observations\label{tab:newopt}}

\tablehead{ 

\colhead{} & \colhead{} &  \colhead{} & \colhead{Obs Date}  & \colhead{Optical} & \colhead{Sp. Type} & \colhead{Other} \\

\colhead{Object} & \colhead{Telescope} & \colhead{Instrument} &  \colhead{(UT)} & \colhead{Sp. Type\tablenotemark{b}} & \colhead{Ref.} & \colhead{Refs.\tablenotemark{c}} 
}
\startdata     
00325584-4405058  &   Keck I	& LRIS &	2009 Oct 11 & L0$\gamma$	& \cite{Cruz09_lowg} &	\cite{Reid08}             \\
00550564+0134365  &   Keck I	& LRIS &	2009 Oct 11 & L2$\gamma$	& This paper &	              \\
0117474-340325    &   Keck I	& LRIS &	2009 Oct 12 & L1$\beta$	& \cite{Cruz03} &	\cite{Cruz03}	            \\
%0117474-340325    &   Keck I	& LRIS &	2009 Oct 12 & L1$\beta$	& This paper &	Cruz03              \\
02103857-3015313  &   Keck I	& LRIS &	2009 Oct 11 & L0$\gamma$	& This paper &	             \\
0241115-032658    &   Keck I	& LRIS &	2009 Oct 12 & L0$\gamma$	& \cite{Cruz09_lowg} &	\cite{Cruz07}              \\
%0436278-411446    &  Keck I	& LRIS &	2009 Oct 11 & M8β	& \cite{Cruz07} &	NN5                 \\
%0436278-411446    &  Keck I	& LRIS &	2009 Oct 11 & M8β	& This paper &	NN5                 \\
%0512063-294954    &   GS		& GMOS &	2005 Oct 10 & L5$\gamma$	& Kirkpatrick08	& Cruz03        \\
0518461-275645    &   Keck I	& LRIS &	2009 Feb 17  & L1$\gamma$	& This paper	& \cite{Cruz07}            \\
%05341594-0631397  &  Keck I	& LRIS &	2009 Oct 11 & M8γ	& Kirkpatrick10 &	Kirkpatrick1\\  
0536199-192039    &   Keck I	& LRIS &	2009 Feb 17  & L2$\gamma$	& This paper	& \cite{Cruz07}            \\
13313310+3407583  &   Keck I	& LRIS &	2009 Feb 17  & L0	& \cite{Reid08}	& \cite{Reid08}           \\
15382417-1953116  &   Keck I	& LRIS &	2010 Jul 17  & L4$\gamma$	& This paper	&            \\
15515237+0941148  &   Keck I	& LRIS &	2010 Jun 18  & L3.5$\gamma$	& This paper	& \cite{Reid08}           \\
\nodata  &   Keck I	& LRIS &	2009 Feb 17  & L3.5$\gamma$	& This paper	& \cite{Reid08}           \\
1615425+495321    &   Keck I	& LRIS &	2010 Jun 18  & L4$\gamma$	& This paper	& \cite{Cruz07}            \\
\nodata    &   Keck I	& LRIS &	2009 Feb 17  & L4$\gamma$	& This paper	& \cite{Cruz07}            \\
1711135+232633    &   Keck I	& LRIS &	2010 Jun 18  & L0$\gamma$	& This paper	& \cite{Cruz07}            \\
1726000+153819    &   Keck I	& LRIS &	2010 Jun 18  & L3.5$\gamma$ &	\cite{Cruz09_lowg}	& \cite{K00}               \\
%19355595-2846343  &  Keck I	& LRIS &	2009 10 11 & M9γ	& This paper	& \cite{Reid08}           \\
%2002507-052152    &   Keck I	& LRIS &	2010 Jun 18  & L5$\beta$	& This paper	& \cite{Cruz07}            \\
20343769+0827009  &   Clay 		& LDSS-3 &  2008 Oct 07  & L1	& This paper	&             \\
21373742+0808463   & Keck I	& LRIS & 2010 Jul 17 & L5 & This paper & \cite{Reid08} \\	
2154345-105530    &   Keck I	& LRIS &	2009 Oct 11 & L4$\beta$	& This paper &	              \\
23153135+0617146  &  Keck I 	& LRIS &	2009 Oct 11 & L0$\gamma$ &	This paper &		        \\
\enddata

\tablecomments{Spectra shown in Figure~\ref{fig:newopt}.}

% \tablenotetext{a}{The sexagesimal right ascension and declination suffix of the full 2MASS All-Sky Data Release designation 
% (2MASS Jhhmmss[.]$\pm$ssddmmss[.]s) is listed for each object.
% The coordinates are given for the J2000.0 equinox; the units of right ascension are hours, minutes, and seconds; 
% and units of declination are degrees, arcminutes, and arcseconds.}
% 
% \tablenotetext{c}{Uncertainties on spectral types are $\pm$ 0.5 subtypes except where noted by one or two colons, indicating an uncertainty of $\pm$1 and $\pm$2 types respectively.}
% 
 \tablenotetext{c}{References listed pertain to discovery and previous \emph{optical} classifications.} 
% %See \citet{Reid06_binary} for objects targeted with NICMOS but unresolved.}
% 
% \tablenotetext{d}{Distance estimate based on trigonometric parallax.}
% 

\end{deluxetable}



\subsection{Near-Infrared Spectra}

In addition to a far-red optical spectrum, for an L dwarf to be included in the optical-NIR sample, it is required to have a spectrum obtained with the SpeX spectrograph on the 3~m NASA Infrared Telescope Facility using the low-resolution prism mode \citep{Spex}.
SpeX prism mode provides a wavelength coverage of 0.8--2.5~$\micron$ in a single order and an average resolving power of $\lambda/\Delta\lambda\sim120$.
SpeX has proven to be a popular workhorse instrument for brown dwarf spectroscopic follow-up and, as a result, there is a plethora of data publicly available in SpeX Prism Library Analysis Toolkit (SPLAT)\footnote{SpeX Prism Library Analysis Toolkit (SPLAT): \url{http://www.browndwarfs.org/splat}} and its predecessor, the SpeX Prism Library \footnote{SpeX Prism Spectral Library: \url{http://pono.ucsd.edu/~adam/browndwarfs/spexprism/}} \citep{Burgasser:2014tr}.

\label{sec:obs_new_nir}
Included in the sample are \NewPrismSpectra~new SpeX prism spectra of \NewPrismObjects~L dwarfs.
Observations were obtained over 37 nights during 2003--2011. The targets and observation dates are listed in Table~\ref{tab:newnir}.
The targets were observed as dithered pairs (ABBA) to enable pair-wise sky subtraction. A0 stars were observed at a similar airmass and time as the target to provide both telluric and flux calibration.
The data were flat-fielded, extracted, wavelength calibrated, and telluric corrected with Spextool \citep{Cushing04,Spextool2}.
In the spectral type distribution of the entire optical-NIR sample shown in Figure~\ref{fig:spthist}, objects with new NIR SpeX Prism data are indicated with right hatching.
These data are available \added{as Data behind the Figure} and from SPLAT and the BDNYC Database.

In Appendix~\ref{sec:galaxies}, we show the NIR spectra of objects that were revealed \replaced{to not be}{not to be} ultracool dwarfs and instead are suspected to be galaxies.

% new nir prism data
%!TEX root = /Users/kelle/Dropbox/Pubs IN PROGRESS/Paper XII NIR DATA/Cruz-Nunez/NIRLdwarfs.tex
\begin{deluxetable}{ll}
%\tabletypesize{\scriptsize}
%\tabletypesize{\tiny}
\tablecolumns{2}
\tablecaption{New Near-Infrared IRTF SpeX Prism Observations\label{tab:newnir}}
\tablehead{
\colhead{2MASS} & \colhead{Obs Date (UT)}   \\
\colhead{Designation} &  \colhead{yyyy/mm/dd}
}
\startdata
%00100009$-$2031122 & 2008/07/14 \\
00250365+4759191 & 2003/09/03 \\
00325584$-$4405058 & 2007/11/13 \\
00325584$-$4405058 & 2008/09/07 \\
00332386$-$1521309 & 2003/09/04 \\
\enddata
\tablecomments{Data are publicly available from SPLAT and the BDNYC Database. This table is published in its entirety in a machine readable format in the electronic edition of the journal. A portion is shown here for guidance regarding its form and content.}
\end{deluxetable}


\subsection{Sample Properties}

We made no specific effort to ensure the optical-NIR sample is complete in any dimension, but we did want it to be as representative as possible of the known L~dwarf population.
Because of the known red bias of the photometrically selected 2MASS-based samples \cite[Figure 3]{Schmidt10}, we made a concerted effort to identify L dwarfs in the literature with $J-K_S$ colors bluer than the range of colors spanned by the 2MASS-selected sample and with both optical and SpeX Prism spectra.
This added less than ten objects to the sample.

We did not include all possible low-gravity objects in the optical-NIR sample.
(As described in \citealt{Cruz09_lowg}, $\gamma$ and $\beta$ spectra type suffixes are used to indicate very low-gravity and intermediate-gravity spectral features, respectively.)
For $\beta$~type objects, we only considered L0--L1$\beta$-type objects since the later types do not have more than one object at each spectral type.
For $\gamma$~type objects, we limited our analysis to L0--L4$\gamma$.
The later $\gamma$~type objects (e.g., \objectname[2MASS J03552337+1133437]{2M0355+1133}, \citealt{Cruz09_lowg}) display significant spectral diversity and we defer an analysis of these data to a future paper.

The resulting optical-NIR sample of \sample~objects includes \optField~field-gravity objects and \optLowG~low-gravity objects.
It also includes \NewOptSpectra~new optical spectra for \NewOptObjects~objects and \NewPrismSpectra~new SpeX prism spectra for \NewPrismObjects~objects.
These new data are described in the following subsections and listed in Tables~\ref{tab:newopt} and~\ref{tab:newnir}. The full sample used for this paper is collectively listed in Tables~\ref{tab:field_template}--\ref{tab:lowg_excluded} based on analysis described in \S~\ref{sec:templates}.

The spectral type distribution of the optical-NIR sample is shown in Figure~\ref{fig:spthist}.
The spectral types adopted for objects in the optical-NIR sample are all based on high signal-to-noise ratio ($\gtrsim$50) optical spectra and the typing schemes of \cite{K99} and \cite{Cruz09_lowg}.

Many of the objects in the optical-NIR sample have existing NIR spectral types in the literature; however, these types are based on several different schemes with inconsistent suffix usage.
In order to give a general sense of the pre-existing NIR types for objects in the optical-NIR sample, we have identified a subsample composed of 78 objects in the optical-NIR sample included in the SpeX Prism Library Analysis Toolkit (SPLAT) with a listed NIR spectral type.
(The types originate from 17 different papers but they are predominately from \citet{Knapp04}, \citet{Allers:2010cg}, \citet{Kirkpatrick10}, and \citet{Gagne:2015dc}.)
In Figure~\ref{fig:opt_lit_nir_comparison}, we show these NIR types compared to the optical types and indicate agreement/disagreement between suffixes.
In the 21 cases where the optical and literature types disagree by more than one subtype (falling outside the shaded region), it is much more common for the literature type to be later than the optical compared to the other way around (17/4).
Further, in 26 cases, a full one third of this comparison subsample, the optical and the NIR suffix do not agree.


The $J-K_S$ color distribution of the optical-NIR sample is shown in Figure~\ref{fig:JK_colors}.
All of the objects have reliable 2MASS photometry with uncertainties smaller than 0.1~mag.
The color properties of the field and low-gravity samples are discussed in more detail in \S\S~\ref{sec:templates_normal} and~\ref{sec:templates_lowg}.

\begin{figure}
    \plotone{histogram.pdf}
    \caption{Spectral type distribution of the optical-NIR sample distinguishing field-gravity objects in black and gray and low-gravity objects in orange and red.
    Right hatching indicates objects with new optical spectra and left hatching indicates new NIR spectra; these objects are listed in Tables~\ref{tab:newopt} and \ref{tab:newnir}, respectively.
    Objects included in the field-gravity templates are indicated in black and objects included in the low-gravity templates are indicated in red; these objects are listed in Tables~\ref{tab:field_template} and \ref{tab:lowg_template}, respectively.
    Field-gravity objects that are excluded from the field-gravity templates are indicated in gray and low-gravity objects excluded from the low-gravity templates are indicated in orange; these objects are listed in Tables~\ref{tab:field_excluded} and \ref{tab:lowg_excluded}, respectively.}
  \label{fig:spthist}
\end{figure}

\begin{figure}
  \plotone{lit_field_obj.pdf}
  \caption{Swarm plot representation of optical spectral types compared to literature NIR types from SPLAT for 78 objects in the optical-NIR sample.
  Filled orange, blue, and pink circles indicate objects where the suffixes agree while a black star indicates suffix disagreement.}
  \label{fig:opt_lit_nir_comparison}
\end{figure}

\begin{figure}
    \plotone{JK.pdf}
    \caption{Box plot representation of the 2MASS $J-K_S$ color distribution for the L dwarfs in the optical-NIR sample.
    Quartile ranges (\emph{box and whiskers}) and mean colors (\emph{solid circles}) are shown for the objects included in the field-gravity templates (\emph{black}) and low-gravity templates (\emph{red}).
    The boxes indicates the interquartile range which contains half of the data, the median is indicated with a horizontal line inside the box, and the vertical lines (whiskers) extend to the minimum and maximum of the distribution.
        For bins with five or fewer objects, only whiskers are shown indicating the range of the distribution.
        The number of objects used in calculating the quartiles are indicated above the end caps of the whiskers.
    The median $J-K_S$ uncertainty of all objects included in the templates is shown with a gray error bar.
    Objects excluded from the templates are marked individually (\emph{crosses}, field g: \emph{gray}, low-g: \emph{orange}).
    The properties of the field and low-gravity template samples are summarized in Table~\ref{tab:sample_prop} and~\ref{tab:lg_sample_prop}, respectively. The individual objects included in the field and low-gravity templates are listed in Tables~\ref{tab:field_template} and \ref{tab:lowg_template}, while the excluded objects are listed in Tables~\ref{tab:field_excluded} and~\ref{tab:lowg_excluded}.}
  \label{fig:JK_colors}
\end{figure}
\clearpage

\section{Optically-anchored NIR Spectral Analysis}

\subsection{Method for Normalizing NIR Spectra}
\label{sec:method}
L dwarfs exhibit a wide range of $J-K_S$ colors at each optical spectral type \cite[e.g.,][]{Kirkpatrick08, Schmidt10, Faherty:2012cy} and this color spread poses challenges to the analysis of NIR spectra.
\replaced{Common practice is to normalize }{Most previous analyses have normalized } NIR spectra of L~dwarfs at the \emph{zJ}-band peak (1.28--1.32~$\micron$, e.g., K10).
In the top panel of Figure~\ref{fig:L1fan}, we show the 0.8--2.4~$\micron$ spectral range for 17 field-gravity, optically-typed L1 dwarfs normalized in this manner.
Despite all of the objects having the same optical spectral type and appearing to have similar NIR spectral morphology, the spectra `fan out' at longer wavelengths.
That is, while the spectra lie on top of each other near the normalization point, they gradually diverge from one another at both shorter and longer wavelengths, as would be expected for objects having a range of $J-K_S$ colors.
This fanning effect makes analyzing the similarities or differences in the spectral shapes and absorption features difficult beyond where the spectra are normalized.
K10 attempts to get around this by focusing analysis on the 0.9--1.4$\micron$ region, but that method does not take full advantage of the spectral morphology in the \emph{H} and \emph{K} bands.

As was pointed out over a decade ago by \citet{Leggett:2003tm}, despite the different spectral slopes, when the \emph{H} and \emph{K} bands of objects of the same optical spectral type are separately scaled and overlaid, the agreement between the spectral morphologies is excellent.
In order to directly compare the absorption features in L~dwarf spectra, we utilize this plotting method where each NIR band is normalized separately, using most of the wavelength range of each band.
Specifically, we use the following normalization ranges, which exclude the low signal-to-noise regions in the water bands, to divide the spectra into three sections:
\begin{itemize}
\item 0.87--1.39~$\micron$ for the \emph{zJ}~band,
\item 1.41--1.89~$\micron$ for the \emph{H}~band, and
\item 1.91--2.39~$\micron$ for the \emph{K}~band.
\end{itemize}
These three regions were chosen because they are naturally separated by the strong water absorption bands present in both L~dwarfs' and Earth's atmospheres.
(This method has also been used for T dwarf classification \citep{Burgasser06}.)
Once normalized in this manner, the effect of the color term is substantially reduced and it becomes evident that the low-resolution spectral morphologies among normal L~dwarfs of the same optical spectral type are very similar.
The same objects shown in the top panel of Figure~\ref{fig:L1fan} that fan out over the 0.8--2.4~$\micron$ range are shown in the bottom panel plotted band-by-band, using the above defined normalization ranges for each panel.
Not only do the spectral similarities between the objects stand out more clearly in this panel visualization, it is also easier to distinguish truly peculiar (e.g, `red' and `blue') spectra from the bulk population.

This three component normalization method is further justified because it can be used to analyze data obtained by spectrographs which obtain spectra one band at a time and where full $JHK$ spectral coverage is not ensured.
Two such instruments which are used to study brown dwarfs and exoplanets are the Gemini Planet Imager (GPI) and the Spectro-Polarimetric High-contrast Exoplanet Research instrument (SPHERE) \citep{Beuzit:2008gt,Macintosh:2014js}.
For example, \citet{Macintosh:2015ew} only obtained $JH$ spectra for 51 Eri b.

With this plotting method, the color term present in the NIR spectra is approximated as a three component piece-wise step function.
Since the reddening contributing to the color term has been found to be well described by a smooth function \citep{Marocco:2014kr,Hiranaka:2016jm}, there is likely still a small amount of residual reddening remaining within each band.
\replaced{One could imagine trying to de-redden each band using a reddening curve.
However, this would require making physical assumptions about the cause of the reddening (e.g, grain size distribution) in order to derive a reddening curve for each object.
Not only would this approach be onerous, it also violates one of the precepts of the MK System for spectral classification \citep{Morgan:1984wy,Cushing:2014un}.}{We did not attempt to normalize using a smooth function.}

\begin{figure}
  \epsscale{0.90}
  \plotone{L1F_fan.pdf}
  \plotone{L1F.pdf}
  \caption{\emph{Top}: NIR spectral range of L1 type dwarfs normalized using the peak of the \emph{zJ} band (1.28--1.32~$\micron$). The fanning out of the spectra to the right and left of the normalizing section makes spectral comparison difficult.
    \emph{Bottom}: Same data as in top panel but normalized and plotted band-by-band. The spectra are normalized individually by band (optical, \emph{zJ}, \emph{H}, and \emph{K} bands) using nearly the entire band range as defined in \S~\ref{sec:method}. Prominent atomic and molecular features are indicated, as well as telluric correction artifacts due to poor line scaling of the  flux-calibrating A0 star's Hydrogen lines ($\earth$).}
  \label{fig:L1fan}
\end{figure}

\subsection{Optically-anchored NIR spectral average templates}
\label{sec:templates}

Spectral average templates are a common tool used in stellar and extragalactic spectroscopy to reveal the most common spectral morphologies, average out defects of individual observations, and to provide representative data with high signal-to-noise ratios.
SDSS spectra have been used to make M and L dwarf templates in the far-red optical wavelength regime \citep{Bochanski07_templates, Schmidt:2014jc} but no set of templates have been defined for the NIR spectra of L dwarfs.
In this section, we use the optical-NIR sample of L dwarf spectra to generate spectral average templates which are anchored to the optical classification and accurately reflect both the typical spectral morphology of L dwarfs and the variations of their spectral morphology.

Using the sample and data described in \S~\ref{sec:sample}, we have combined the NIR spectra of objects of the same optical spectral type to make NIR spectral average templates for field and low-gravity L~dwarfs at each integer spectral type, \added{in the range} L0--L8. Objects with half-integer spectral types were rounded down to the integer type.
Since our analysis is anchored in the optical, we do not have an L9, which has only been defined based on NIR data \citep{K99,Geballe02}.
Objects were included in this analysis as long as their optical spectra could be classified on either the \citet{K99} system for field-gravity spectra or the \citet{Cruz09_lowg} system for low-gravity spectra; known binarity and/or NIR spectral peculiarity was not taken into account.
Below, we describe the methods used to construct the templates.

%three samples
We divided the optical-NIR sample (\sample) into three subsamples based on their optical spectral morphology:
\begin{enumerate}
	\item a field-gravity sample with \optField~objects which have spectra similar to the \citet{K99} field spectral standards.
	\item a low-gravity $\gamma$ sample with \optGamma~objects with very strong low-gravity features as described in \citet{Cruz09_lowg}.
	\item a low-gravity $\beta$ sample with \optBeta~objects with intermediate-strength low-gravity features as described in \citet{Cruz09_lowg}.
\end{enumerate}

All of the NIR spectra of the same optical spectral type were combined to make the first iteration of each template.
The spectra were normalized band-by-band and combined using the uncertainty weighted average flux at each pixel.
A \replaced{reduced $\chi^2$-like metric }{minimization metric, $M$,} was calculated comparing each observed spectrum to the template:
\begin{equation}
M=\frac{\sum_{\lambda_i=1}^{n} \frac{[F_{obs}(\lambda_i) - F_{template}(\lambda_i)]^2}{\sigma(\lambda_i)^2}}{n-1}
\end{equation}
where $\lambda_i$ is each wavelength element, $n$ is the total number of wavelength elements in the band, $F_{obs}$ is the normalized flux of the spectrum, $F_{template}$ is the flux of the template, and $\sigma$ is the standard deviation of the fluxes used to make the template. (The uncertainty in the observed spectrum was not used.)
Spectra with \replaced{reduced $\chi^2 > 2$}{$M > 2$} in any band were tagged for exclusion.
The second iteration of the template was calculated with the remaining spectra and the \replaced{$\chi^2$}{$M$} values were re-calculated using the new template.
This process was iterated 3--6 times until no additional objects were rejected and the \replaced{reduced $\chi^2$}{$M$} values stabilized. The code used to make the templates is available on GitHub\footnote{Code used to generate templates: \url{https://github.com/kelle/Make_NIRTemplates}.}.

The average \replaced{$\chi^2$}{$M$} values of the spectra included in the templates are $\sim0.9$.
We found that a \replaced{$\chi^2$}{$M$} cutoff of two balances our desire to maximize the number of spectra used to define the templates while still excluding spectra with substantially peculiar spectral morphologies.
%details for each subtype where necessary
While the \replaced{$\chi^2 > 2$}{$M > 2$} criterion was appropriate for most subtypes, a few required adjusting.
For the field-gravity L1~template, a \replaced{$\chi^2$}{$M$} limit of two excluded mostly bluer objects and does not result in a representative sample of field L1~objects being included in the template.
We instead used a \replaced{$\chi^2 > 2.3$}{$M > 2.3$} criterion which results in a template that more accurately reflects the range of spectral morphologies included in the L1~subtype.
\textbf{For L7 and L8 field-gravity templates, where the number of objects are small, a tighter criterion of \replaced{$\chi^2 > 1.4$}{$M > 1.4$} was used because otherwise no objects are excluded.}

This iterative process resulted in 127 objects being included in 16 spectral average templates (Tables~\ref{tab:field_template} and \ref{tab:lowg_template}) and 44 being excluded (Tables~\ref{tab:field_excluded} and \ref{tab:lowg_excluded}).
The resulting spectral average templates of field-gravity objects are shown in Figure~\ref{fig:field_templates} and discussed in \S~\ref{sec:fieldg}; the low-gravity templates are in Figure Sets~\ref{fig:beta_templates} and~\ref{fig:gamma_templates} and discussed in \S~\ref{sec:lowg}; and spectra excluded from the templates are displayed in Figures~\ref{fig:field_excluded} and \ref{fig:lowg_excluded}.
\added{The templates are available as the Data behind the Figures and from GitHub\footnote{Templates: \url{https://github.com/BDNYC/NIRTemplates}}.}
Shading is used to indicate the 1$\sigma$ variance (dark gray) and range of flux values (light gray) used to calculate the template at each pixel.
We also show spectral averages of the optical spectra along with the NIR templates. The optical spectra were smoothed to approximately match the resolution of the NIR spectra and light gray shading indicates the range of flux values used at each pixel.
\deleted{The templates are available from GitHub.}

Both the normalization method and the spectral templates described greatly facilitate the comparative analysis of the NIR spectral morphology of L dwarfs.
As more optical and/or NIR data are collected for L dwarfs, the templates could potentially be updated.
In anticipation of this, we dub the templates presented here version 1.0.

\begin{figure}
  \epsscale{0.9}
  \plotone{L0strip_f.pdf}
  \plotone{L1strip_f.pdf}
  \plotone{L2strip_f.pdf}
  \caption{Spectral average templates for field-gravity objects. The average is shown as a solid black line. At each pixel, the 1$\sigma$ variance is shown as dark gray shading and range of flux values are indicated by light gray. Objects included in the templates are listed on the left in order of increasing $J-K_S$ color. The most prominent atomic and molecular features are indicated. \added{The templates are available as Data behind the Figure.}}
  \label{fig:field_templates}
\end{figure}

\begin{figure}
  \figurenum{6.2}
  \plotone{L3strip_f.pdf}
  \plotone{L4strip_f.pdf}
  \plotone{L5strip_f.pdf}
  \caption{Field templates continued.}
\end{figure}

\begin{figure}
  \figurenum{6.3}
  \plotone{L6strip_f.pdf}
  \plotone{L7strip_f.pdf}
  \plotone{L8strip_f.pdf}
  \caption{Field templates continued.}
\end{figure}

\begin{figure}
  \epsscale{0.9}
  \plotone{L0strip_f_excluded.pdf}
  \plotone{L1strip_f_excluded.pdf}
  \plotone{L2strip_f_excluded.pdf}
  \caption{Spectra excluded from the field-gravity templates (colors). The spectral average template is shown as a solid black line. Objects excluded from the templates are listed on the left in order of increasing $J-K_S$ color. The most prominent atomic and molecular features are indicated.}
  \label{fig:field_excluded}
\end{figure}

\begin{figure}
  \figurenum{7.2}
  \plotone{L3strip_f_excluded.pdf}
  \plotone{L4strip_f_excluded.pdf}
  \plotone{L5strip_f_excluded.pdf}
  \caption{Excluded field object continued.}
\end{figure}

\begin{figure}
  \figurenum{7.3}
  \plotone{L6strip_f_excluded.pdf}
  \plotone{L7strip_f_excluded.pdf}
  \plotone{L8strip_f_excluded.pdf}
  \caption{Excluded field objects continued.}
\end{figure}

% included in normal template
\clearpage
%!TEX root = /Users/kelle/Dropbox/Pubs IN PROGRESS/Paper XII NIR DATA/Cruz-Nunez/NIRLdwarfs.tex
\begin{deluxetable}{lllllll}
\tabletypesize{\tiny}
%\rotate
\tablewidth{0pt}
\tablecolumns{7}
\tablecaption{L Dwarfs Included in the Field Gravity Template Sample\label{tab:field_template}}
\tablehead{  \colhead{} &  \multicolumn{2}{c}{2MASS} \\
\cline{2-3} & \colhead{}  & \colhead{} & \colhead{Optical} & \colhead{Sp. Type} & \colhead{SpeX Prism}  & \colhead{Other} \\
\colhead{Object} & \colhead{$J$} &  \colhead{$J-K_S$}  &  \colhead{Sp. Type} & \colhead{Ref.} & \colhead{Ref.} & \colhead{Refs.\tablenotemark{a}}
}
\startdata
0230155+270406	 & 						14.294$\pm$0.027	& 1.308$\pm$0.035	& L0:	& Cruz07	& This Paper	& Cruz07	This Paper, Cruz07, Cruz07                         \\
0239424-173547	 & 						14.291$\pm$0.032	& 1.252$\pm$0.045	& L0	& Cruz03	& This Paper	& Cruz03	This Paper, Cruz03, Cruz03                         \\
03140344+1603056	 & 					12.526$\pm$0.024	& 1.288$\pm$0.032	& L0	& paper10	& This Paper	& paper10	This Paper, paper10, paper10                       \\
03454316+2540233		 & 13.997$\pm$0.027 	& 1.325$\pm$0.036	& L0		& K99		& This Paper	& K99                       This Paper, K99, K99                           \\
DENIS-P 0652-2534 & 					12.759$\pm$0.023	& 1.243$\pm$0.031	& L0	& phanbao08	& This Paper	& PhanBao08	This Paper, phanbao08, PhanBao08                   \\
0746425+200032	 & 						11.759$\pm$0.02		& 1.291$\pm$0.03	& L0.5	& R00	    &			&                       	                              \\
10221489+4114266 HD 89744B & 			14.9  $\pm$0.04		& 1.292$\pm$0.056	& L0	& Wilson01	& Burgasser080320	& Wilson01	Burgasser080320, Wilson01, Wilson01   \\
12321827-0951502	 & 					13.727$\pm$0.027	& 1.173$\pm$0.04	& L0	& Paper10	& This Paper	& Paper10	This Paper, Paper10, Paper10                       \\
1404449+463429	 & 						14.338$\pm$0.029	& 1.281$\pm$0.042	& L0:	& Cruz07	& This Paper	& Cruz07	This Paper, Cruz07, Cruz07                         \\
14213145+1827407	 & 					13.231$\pm$0.021	& 1.288$\pm$0.029	& L0	& Paper10	& This Paper	& NN	This Paper, Paper10, NN                                \\
SDSSp J163600.79-003452.6 & 			14.59 $\pm$0.044	& 1.175$\pm$0.057	& L0	& Fan00	& This Paper	& Fan00	This Paper, Fan00, Fan00                                   \\
1707333+430130	 & 						13.974$\pm$0.026	& 1.35 $\pm$0.04	& L0.5pec &	Cruz03	& This Paper	& Cruz03	                                               \\
SSSPM 23101853-1759094 & 				14.376$\pm$0.032	& 1.407$\pm$0.041	& L0:	& Cruz07	& This Paper	& Lodieu02	This Paper, Cruz07, Lodieu02                       \\
23312938+1552228	 & 					15.057$\pm$0.041	& 1.057$\pm$0.079	& L0	& West08	& This Paper	& West08	This Paper, West08, West08                         \\
\hline
0141032+180450	 & 						13.875$\pm$0.025	& 1.383$\pm$0.038	& L1	& Cruz07	& This Paper	& Wilson02	This Paper, Cruz07, Wilson02                       \\
SSSPM J0219-1939		 & 14.11 $\pm$0.027 	& 1.2  $\pm$0.043	& L1		& Cruz03	& This Paper	& Lodieu02                  This Paper, Cruz03, Lodieu02                   \\
0213288+444445			 & 13.494$\pm$0.025 	& 1.281$\pm$0.034	& L1.5pec	& Cruz03	& NIRYoung	& Cruz03                                                               \\
02355993-2331205	GJ 1048B & 			13.67 $\pm$0.15	& 1.484$\pm$0.17		& L1	&			&	Burgasser080320 &	 	Burgasser080320, davy,                                \\
05431887+6422528	 & 					13.567$\pm$0.031	& 1.517$\pm$0.039	& L1	 & Paper10	& This Paper	& Paper10	This Paper, Paper10, Paper10                       \\
06022216+6336391	 & 					14.266$\pm$0.03	& 1.583$\pm$0.038		& L1:	 & paper10	& This Paper	& Paper10	This Paper, paper10, Paper10                       \\
07231462+5727081	 & 					13.97 $\pm$0.026	& 1.357$\pm$0.04	& L1	 & Paper10	& This Paper	& Paper10	This Paper, Paper10, Paper10                       \\
DENIS-P 0751-2530		 & 13.161$\pm$0.024 	& 1.17 $\pm$0.034	& L1.5		& phanbao08	& This Paper	& PhanBao08                 This Paper, phanbao08, PhanBao08               \\
SDSS J104524.00-014957.6 & 				13.16 $\pm$0.024	& 1.38 $\pm$0.033	& L1	 & Cruz07	& This Paper	& Hawley02	This Paper, Cruz07, Hawley02                       \\
SDSS J104842.81+011158.2 & 				12.924$\pm$0.023	& 1.301$\pm$0.033	& L1	 & paper10	& This Paper	& Hawley02	This Paper, paper10, Hawley02                      \\
14392836+1929149		 & 12.759$\pm$0.019 	& 1.213$\pm$0.029	& L1		& K99		& Burgasser04T	& K99                   Burgasser04T, K99, K99                    \\
1807159+501531	 & 						12.934$\pm$0.024	& 1.332$\pm$0.035	& L1.5	 & Cruz03	& This Paper	& Wilson02	This Paper, Cruz03, Wilson02                       \\
19223062+6610194		 & 14.571$\pm$0.032 	& 1.409$\pm$0.05	& L1		& Paper10	& This Paper	& Paper10                   This Paper, Paper10, Paper10                   \\
20343769+0827009	 & 					14.464$\pm$0.034	& 1.384$\pm$0.046	& L1	 & This Paper	& This Paper	&                           \\
2057540-025230			 & 13.121$\pm$0.024 	& 1.397$\pm$0.035	& L1.5:		& cruz03	& This Paper	& Kendall04                 This Paper, cruz03, Kendall04                  \\
21304464-0845205	 & 					14.137$\pm$0.032	& 1.322$\pm$0.046	& L1	 & Paper10	& Kirkpatrick10	& Kirkpatrick08	Kirkpatrick10, Paper10, Kirkpatrick08  \\
2238074+435317	 & 						13.84 $\pm$0.027	& 1.32 $\pm$0.038	& L1.5	 & Cruz03	& This Paper	& Cruz03	This Paper, Cruz03, Cruz03                         \\
2330225-034718	 & 						14.475$\pm$0.03	& 1.354$\pm$0.045		& L1:	 & Cruz07	& This Paper	& Cruz07	This Paper, Cruz07, Cruz07                         \\
\hline
00154476+3516026	 & 					13.878$\pm$0.03	& 1.614$\pm$0.038		& L2	 & K00	& adam	K00	    &                                                          \\
0241536-124106	 & 						15.605$\pm$0.068	& 1.674$\pm$0.093	& L2:	 & Cruz03	& Burgasser080320	& Cruz03	                                           \\
0408290-145033	 & 						14.222$\pm$0.03	& 1.405$\pm$0.038		& L2	 & cruz03	& This Paper	& Wilson02	This Paper, cruz03, Wilson02                       \\
0523382-140302	 & 						13.084$\pm$0.024	& 1.446$\pm$0.036	& L2.5	 & Cruz03	& This Paper	& Wilson02	This Paper, Cruz03, Wilson02                       \\
08230838+6125208	 & 					14.82 $\pm$0.036	& 1.624$\pm$0.046	& L2:	 & paper10	& This Paper	& Paper10	This Paper, paper10, Paper10                       \\
08355829+0548308	 & 					14.533$\pm$0.036	& 1.365$\pm$0.05	& L2:	 & paper10	& This Paper	& Paper10	This Paper, paper10, Paper10                       \\
14090310-3357565	 & 					14.248$\pm$0.026	& 1.383$\pm$0.039	& L2	 & kirkpatrick08	& This Paper	& Paper10	This Paper, kirkpatrick08, Paper10             \\
1430435+291540	 & 						14.273$\pm$0.028	& 1.502$\pm$0.038	& L2	 & Cruz03	& This Paper	& Cruz03	                                               \\
16573454+1054233	 & 					14.15 $\pm$0.036	& 1.349$\pm$0.047	& L2	 & Paper10	& This Paper	& Paper10	This Paper, Paper10, Paper10                       \\
SDSS J202820.32+005226.5 & 				14.298$\pm$0.035	& 1.505$\pm$0.046	& L2	 & Hawley02	& Burgasser04t	& Hawley02	Burgasser04t, Hawley02, Hawley02          \\
2041428-350644	 & 						14.887$\pm$0.033	& 1.486$\pm$0.05	& L2:	 & Cruz07	& This Paper	& Cruz07	This Paper, Cruz07, Cruz07                         \\
2104149-103736	 & 						13.841$\pm$0.029	& 1.472$\pm$0.038	& L2.5	 & Cruz03	& This Paper	& cruz03	This Paper, Cruz03, cruz03                         \\
21371044+1450475	 & 					14.133$\pm$0.03	& 1.318$\pm$0.045		& L2	 & Paper10	& This Paper	& Paper10	This Paper, Paper10, Paper10                       \\
\hline
00345684-0706013	 & 					15.531$\pm$0.06	& 1.589$\pm$0.088		& L3	 & Kendall03& 	This Paper	& Kendall03	This Paper, Kendall03, Kendall03                   \\
00361617+1821104	 & 					12.466$\pm$0.027	& 1.408$\pm$0.034	& L3.5	 & Cruz07	& This Paper	& K00	                                                   \\
0218291-313322	 & 						14.728$\pm$0.04	& 1.574$\pm$0.053		& L3	 & Cruz03	& This Paper	& Cruz03	This Paper, Cruz03, Cruz03                         \\
0325013+225303	 & 						15.426$\pm$0.049	& 1.652$\pm$0.069	& L3.5	 & Cruz07	& This Paper	& Cruz07	This Paper, Cruz07, Cruz07                         \\
0700366+315726	 & 						12.923$\pm$0.023	& 1.606$\pm$0.033	& L3:	 & Paper10	& NIRyoung	& TK03	                                                   \\
08234818+2428577	 & 					14.986$\pm$0.043	& 1.609$\pm$0.052	& L3	 & Paper10	& Burgasser10spex	& Paper10	Burgasser10spex, Paper10, Paper10     \\
DENIS-P J1058.7-1548 & 					14.155$\pm$0.035	& 1.623$\pm$0.045	& L3	 & k99	& This Paper	& Delfosse97	This Paper, k99, Delfosse97                        \\
11000965+4957470	 & 					15.282$\pm$0.043	& 1.808$\pm$0.054	& L3.5	 & Paper10	& This Paper	& Paper10	This Paper, Paper10, Paper10                       \\
1146344+223052	 & 						14.165$\pm$0.028	& 1.575$\pm$0.038	& L3	 & K99	& burgasser10spex	& K99	                                               \\
12070374-3151298	 & 					15.85 $\pm$0.066	& 1.853$\pm$0.087	& L3:	 & Paper10	& This Paper	& Paper10	This Paper, Paper10, Paper10                       \\
1506544+132106	 & 						13.365$\pm$0.023	& 1.624$\pm$0.03	& L3	 & NN	& This Paper	& NN	This Paper, NN, NN                                         \\
1645220+300407	 & 						15.19 $\pm$0.038	& 1.603$\pm$0.054	& L3	 & Cruz07	& This Paper	& Cruz07	This Paper, Cruz07, Cruz07                         \\
20360316+1051295	 & 					13.95 $\pm$0.026	& 1.503$\pm$0.037	& L3	 & Paper10	& Burgasser10spex	& Paper10	Burgasser10spex, Paper10, Paper10     \\
2242531+254257	 & 						14.812$\pm$0.039	& 1.764$\pm$0.052	& L3	 & Cruz07	& Burgasser10spex	& Gizis03	Burgasser10spex, Cruz07, Gizis03      \\
\hline
0025036+475919	HD 2057B & 				14.84 $\pm$0.038	& 1.938$\pm$0.069	& L4	 & Cruz07	& NIRyoung	& Cruz07	                                               \\
0337036-175807	 & 						15.621$\pm$0.058	& 2.04 $\pm$0.071	& L4.5	 & K00	& This Paper	& K00	This Paper, K00, K00                                       \\
1104012+195921	 & 						14.38 $\pm$0.026	& 1.43 $\pm$0.039	& L4	 & Cruz03	& Burgasser04t	& Cruz03	Burgasser04t, Cruz03, Cruz03                \\
14283132+5923354	 & 					14.781$\pm$0.037	& 1.516$\pm$0.046	& L4	 & Paper10	& This Paper	& Paper10	This Paper, Paper10, Paper10                       \\
18212815+1414010	 & 					13.431$\pm$0.024	& 1.781$\pm$0.032	& L4.5	 & Looper08dirty	& Looper08dirty	& Looper08dirty	Looper08dirty, Looper08dirty, Looper08dirty \\
2151254-244100	 & 						15.752$\pm$0.08	& 2.103$\pm$0.095		& L4\tablenotemark{b}	 & This Paper	& This Paper	&  \cite{Cruz07}                    \\
2158045-155009	 & 						15.04 $\pm$0.04	& 1.855$\pm$0.054		& L4:	 & Cruz07	& Kirkpatrick10	& Cruz07	Kirkpatrick10, Cruz07, Cruz07          \\
23174712-4838501	 & 					15.15 $\pm$0.038	& 1.969$\pm$0.048	& L4pec	 & paper10	& Kirkpatrick10	& Paper10	                                           \\
\hline
00043484-4044058 LHS 102B & 			13.109$\pm$0.024	& 1.713$\pm$0.035	& L5	 & K00	& Burgasser07binaries	& Goldman99	                                   \\
0144353-071614			 & 14.191$\pm$0.026 	& 1.923$\pm$0.035	& L5		& cruz03	& This Paper	& Liebert03                 This Paper, cruz03, Liebert03                  \\
0205034+125142	 & 						15.679$\pm$0.056	& 2.008$\pm$0.067	& L5	 & K00	& This Paper	& K00	This Paper, K00, K00                                       \\
0310140-275645	 & 						15.795$\pm$0.071	& 1.836$\pm$0.094	& L5	 & Cruz07	& This Paper	& Cruz07 This Paper, Cruz07, Cruz07                        \\
03582255-4116060	 & 					15.846$\pm$0.087	& 2.008$\pm$0.1		& L5	 & Paper10	& This Paper	& Paper10 This Paper, Paper10, Paper10                     \\
06244595-4521548	 & 					14.48 $\pm$0.029	& 1.885$\pm$0.039	& L5	 & kelle	& This Paper	& paper10 This Paper, kelle, paper10                       \\
0652307+471034	 & 						13.511$\pm$0.023	& 1.817$\pm$0.03	& L5	 & Cruz07	& This Paper	& Cruz03 This Paper, Cruz07, Cruz03                        \\
0835425-081923	 & 						13.169$\pm$0.024	& 2.033$\pm$0.032	& L5	 & Cruz07	& This Paper	& Cruz03 This Paper, Cruz07, Cruz03                        \\
09054654+5623117	 & 					15.395$\pm$0.052	& 1.665$\pm$0.064	& L5	 & Paper10	& This Paper	& Paper10 This Paper, Paper10, Paper10                     \\
DENIS-P J1228.2-1547AB & 				14.378$\pm$0.03	& 1.611$\pm$0.042		& L5	 & K99	& Burgasser10spex	& Delfosse97	                                       \\
1343167+394508	 & 						16.162$\pm$0.071	& 2.012$\pm$0.086	& L5	 & K00	& This Paper	& K00	This Paper, K00, K00                                   \\
17461199+5034036	 & 					15.096$\pm$0.058	& 1.567$\pm$0.073	& L5blue & 	paper10	& This Paper	& Paper10	                                           \\
21373742+0808463	 & 					14.774$\pm$0.032	& 1.755$\pm$0.042	& L5	& This Paper	& This Paper	& \cite{Reid08}                   \\
22120703+3430351	 & 					16.316$\pm$0.097	& 1.946$\pm$0.119	& L5:	& Paper10	& This Paper	& Paper10	This Paper, Paper10, Paper10                   \\
2249091+320549	 & 						15.482$\pm$0.06	& 1.894$\pm$0.076		& L5	& cruz07	& This Paper	& Cruz07	This Paper, cruz07, Cruz07                     \\
\hline
0103320+193536	 & 						16.288$\pm$0.08	& 2.139$\pm$0.099		& L6	& K00	& This Paper &	K00	                                                       \\
0439010-235308	 & 						14.408$\pm$0.029	& 1.592$\pm$0.037	& L6.5	& cruz03	& This Paper	& Cruz03	This Paper, cruz03, Cruz03                     \\
0850359+105715	 & 						16.465$\pm$0.113	& 1.992$\pm$0.131	& L6	& K99	& NIRyoung &	K99	                                                   \\
09153413+0422045	 & 					14.548$\pm$0.03	& 1.537$\pm$0.051		& L6	& paper10	& NIRyoung &	paper10	                                           \\
1010148-040649	 & 						15.508$\pm$0.059	& 1.889$\pm$0.075	& L6	& cruz07	& This Paper & 	Cruz03	This Paper, cruz07, Cruz03                         \\
11181292-0856106	 & 					15.74 $\pm$0.087	& 1.56 $\pm$0.12	& L6blue & 	Kirkpatrick10	& Kirkpatrick10	& Kirkpatrick10	                           \\
1515009+484739	 & 						14.111$\pm$0.029	& 1.611$\pm$0.038	& L6	& Cruz07	& This Paper	& Wilson02	This Paper, Cruz07, Wilson02                   \\
15261405+2043414	 & 					15.586$\pm$0.055	& 1.664$\pm$0.076	& L6	& K00	& Burgasser04t	& K00	Burgasser04t, K00, K00                      \\
2132114+134158	 & 						15.795$\pm$0.062	& 1.956$\pm$0.085	& L6	& Cruz07	& Siegler07	& Cruz07	                                           \\
\hline
0318540-342129	 & 						15.569$\pm$0.055	& 2.062$\pm$0.067	& L7	& cruz07	& This Paper	& cruz07	This Paper, cruz07, cruz07                                 \\
0825196+211552	 & 						15.1  $\pm$0.034	& 2.072$\pm$0.043	& L7.5	& K00	& This Paper	& K00	This Paper, K00, K00                                               \\
1728114+394859	 & 						15.988$\pm$0.076	& 2.079$\pm$0.09	& L7	& K00	& NIRYoung	& K00	                                                               \\
21522609+0937575	 & 	15.19 $\pm$0.032	& 1.847$\pm$0.047	& L7:\tablenotemark{c}	& This Paper	& NIRYoung	& Paper10	                                                       \\
\hline
DENIS-P J0255-4700 & 					13.246$\pm$0.027	& 1.688$\pm$0.036	& L8	& Kirkpatrick08	& Burgasser06	& M99	Burgasser06, Kirkpatrick08, M99                    \\
02572581-3105523	 & 					14.672$\pm$0.039	& 1.796$\pm$0.05	& L8	& Kirkpatrick08	& This Paper &	Kirkpatrick08	This Paper, Kirkpatrick08, Kirkpatrick08           \\
0328427+2302052	& 						16.693$\pm$0.14	& 1.777$\pm$0.18		& L8	& K00	& Burgasser080320	& K00	                                                       \\
09121469+1459396 Gl 337CD & 			15.51 $\pm$0.08	& 1.467$\pm$0.102		& L8	& Wilson01	& Burgasser10spex	& Wilson01	                                               \\
15232263+3014562 Gl 584C & 				16.06 $\pm$0.1	& 1.712$\pm$0.12		& L8	& K01	& Burgasser10spex &	K01	Burgasser10spex, K01, K01                             \\
2325453+425148	 & 						15.493$\pm$0.05	& 1.729$\pm$0.068		& L8	& Cruz07 &	This Paper &	Cruz07	This Paper, Cruz07, Cruz07                                     \\
\enddata

\tablecomments{Templates generated using spectra of these objects are shown in Figures~\ref{fig:field_templates}.}

\tablenotetext{a}{References listed pertain to discovery and spectral data and are not exhaustive.}
\tablenotetext{b}{Originally typed as L3 in \cite{Cruz07} but reanalysis of same data resulted in a revised type of L4.}
\tablenotetext{c}{Originally typed as L6: in \cite{Reid08} but reanalysis of the same slightly noisy spectrum resulted in a revised type of L7:.}



\end{deluxetable}



% EXCLUDED from normal template
\clearpage
%!TEX root = /Users/kelle/Dropbox/Pubs IN PROGRESS/Paper XII NIR DATA/Cruz-Nunez/NIRLdwarfs_Normal.tex
\begin{deluxetable}{lllllll}
%\tablewidth{0pt}
%\tabletypesize{\footnotesize}
%\tabletypesize{\small}
%\tabletypesize{\scriptsize}
\tabletypesize{\tiny}
\rotate
\tablewidth{0pt}
\tablecolumns{7}
\tablecaption{Normal L Dwarfs Excluded from the Field Gravity Template Sample\label{tab:field_excluded}}

\tablehead{  \colhead{} &  \multicolumn{2}{c}{2MASS} \\

\cline{2-3} & \colhead{}  & \colhead{} &
\colhead{Optical} & \colhead{Sp. Type} & \colhead{SpeX Prism}  & \colhead{Other} \\

\colhead{Object} & \colhead{$J$} &  \colhead{$J-K_S$}  &  \colhead{Sp. Type} & 
\colhead{Ref.} & \colhead{Ref.} & \colhead{Refs.\tablenotemark{a}}
}
\startdata                              
0010001-203112			 & 14.134$\pm$0.024 	& 1.252$\pm$0.038	& L0pec		& Cruz07	& This Paper	& Cruz07                                                               \\
12212770+0257198		 & 13.169$\pm$0.023 	& 1.216$\pm$0.035	& L0		& paper10	& Burgasser080320	& Paper10           Burgasser080320, paper10, Paper10         \\
13313310+3407583		 & 14.333$\pm$0.027 	& 1.448$\pm$0.035	& L0pec		& This Paper	& \cite{Kirkpatrick10} & \cite{Reid08}                                                            \\
SDSS J16585026+1820006	 & 15.484$\pm$0.057 	& 0.912$\pm$0.111	& L0pec		& West08	& This Paper	& West08                                                               \\
SDSS J172244.32+632946.8 & 15.367$\pm$0.051 	& 1.291$\pm$0.088	& L0pec		& Hawley02	& This Paper	& Hawley02                                                             \\
LSPM J1731+2721			 & 12.094$\pm$0.027 	& 1.18 $\pm$0.034	& L0pec		& Paper10	& This Paper	& LSPM-N                                                               \\
23224684-3133231		 & 13.577$\pm$0.027 	& 1.253$\pm$0.036	& L0:		& Paper10	& dagny	& Paper10                                                                  \\
\hline
SIPS 0921-2104			 & 12.779$\pm$0.024 	& 1.089$\pm$0.033	& L1.5		& Paper10	& This Paper	& Deacon05                                                             \\
1300425+191235			 & 12.717$\pm$0.022 	& 1.093$\pm$0.03	& L1blue	& This Paper	& NN                                                                               \\
14403186-1303263		 & 15.379$\pm$0.05  	& 1.128$\pm$0.082	& L1		& Kirkpatrick10	& Kirkpatrick10	& Kirkpatrick10                                                \\
\hline
0445538-304820			 & 13.393$\pm$0.026 	& 1.418$\pm$0.033	& L2pec		& Cruz03	& This Paper	& Cruz03                                                               \\
06160532-4557080		 & 15.157$\pm$0.042 	& 1.559$\pm$0.057	& L2		& paper10	& This Paper	& Paper10                   This Paper, paper10, Paper10                   \\
0847287-153237			 & 13.513$\pm$0.026 	& 1.452$\pm$0.035	& L2pec		& Paper10	& This Paper	& Cruz03                                                               \\
0928397-160312			 & 15.322$\pm$0.043 	& 1.707$\pm$0.067	& L2		& K00		& This Paper	& K00                       This Paper, K00, K00                           \\
1231214+495923			 & 14.617$\pm$0.035 	& 1.482$\pm$0.045	& L2		& cruz07	& This Paper	& Cruz07                    This Paper, cruz07, Cruz07                     \\
13023811+5650212		 & 16.36 $\pm$0.122 	& 1.388$\pm$0.173	& L2		& Kirkpatrick10	& Kirkpatrick10	& Kirkpatrick10                                                \\
1305401-254106 Kelu-1	 & 13.414$\pm$0.026 	& 1.667$\pm$0.035	& L2		& K99		& Burgasser071520	& Ruiz97                                                       \\
\hline
00165953-4056541		 & 15.316$\pm$0.061 	& 1.884$\pm$0.072	& L3.5		& Paper10	& Burgasser10spex	& paper10           Burgasser10spex, Paper10, paper10         \\
00531899-3631102		 & 14.445$\pm$0.026 	& 1.508$\pm$0.039	& L3.5		& Kirkpatrick08	& Burgasser10spex	& Kirkpatrick08 Burgasser10spex, Kirkpatrick08, Kirkpatric08\\
0251148-035245			 & 13.059$\pm$0.027 	& 1.397$\pm$0.033	& L3		& Cruz03	& This Paper	& Wilson02                                                             \\
07171626+5705430		 & 14.636$\pm$0.032 	& 1.691$\pm$0.041	& L3		& Paper10	& Burgasser10spex	& Wilson03          Burgasser10spex, Paper10, Wilson03        \\
DENIS-P J153941.9-052042 & 13.922$\pm$0.029 	& 1.347$\pm$0.041	& L3.5		& Paper10	& adam	& Kendall04                                                                \\
1721039+334415			 & 13.625$\pm$0.023 	& 1.136$\pm$0.03	& L3blue	& Cruz03	& This Paper	& Cruz03                                                               \\
23392527+3507165		 & 15.362$\pm$0.053 	& 1.774$\pm$0.066	& L3.5		& Paper10	& This Paper	& Paper10                   This Paper, Paper10, Paper10                   \\
\hline
SDSSp J033035.13-002534.5 & 			15.311$\pm$0.05	& 1.474$\pm$0.069		& L4	 & Fan00	& This Paper	& Fan00	This Paper, Fan00, Fan00                               \\
11263991-5003550		 & 13.997$\pm$0.032 	& 1.168$\pm$0.043	& L4.5blue	& Burgasser08blue	& Burgasser08blue	& Folkes07  PhanBao08                                  \\
19285196-4356256	 & 					15.199$\pm$0.044	& 1.742$\pm$0.057	& L4	 & Paper10	& Burgasser10spex	& Paper10	Burgasser10spex, Paper10, Paper10 \\
2224438-015852 & 						14.073$\pm$0.027	& 2.051$\pm$0.035	& L4.5	 & K00	& This Paper	& K00	                                                   \\
\hline
0820299+450031			 & 16.279$\pm$0.108 	& 2.061$\pm$0.127	& L5		& K00		& This Paper	& K00                                                                  \\
1507476-162738			 & 12.83 $\pm$0.027 	& 1.518$\pm$0.037	& L5blue	& R00		& This Paper	& R00                                                                  \\
DENIS-P J225210.7-173013 & 14.313$\pm$0.029 	& 1.412$\pm$0.038	& L5		& paper10	& This Paper	& Kendall04                                                            \\
\hline
13314894-0116500		 & 15.46 $\pm$0.04  	& 1.39 $\pm$0.081	& L6blue	& Hawley02	& Burgasser10spex	& Hawley02                                                     \\
21481633+4003594		 & 14.147$\pm$0.029 	& 2.382$\pm$0.037	& L6		& Looper08dirty	& Looper08dirty	& Looper08dirty                                            \\
22443167+2043433		 & 16.476$\pm$0.14  	& 2.454$\pm$0.158	& L6.5		& Kirkpatrick08	& Looper08dirty	& Dahn02                                           \\
\hline
00150206+2959323	 & 					16.158$\pm$0.081	& 1.676$\pm$0.109	& L7blue & 	Kirkpatrick10	& Kirkpatrick10	& Kirkpatrick10	                           \\
SDSS J010752.33+004156.1 & 				15.824$\pm$0.058	& 2.115$\pm$0.073	& L7.5\tablenotemark{b}	& This Paper	& \cite{Burgasser10_spex}	& \cite{Schneider02,Scholz09} \\
DENIS-P J0205.4-1159AB	 & 14.587$\pm$0.03  	& 1.589$\pm$0.042	& L7		& K99		& This Paper	& Delfosse97                                                   \\
0908380+503208	 & 						14.549$\pm$0.023	& 1.604$\pm$0.035	& L7	& Cruz07	& This Paper	& Cruz03	                                                       \\
\hline
SDSS J083008.12+482847.4 & 15.444$\pm$0.048 	& 1.768$\pm$0.061	& L8		& Kirkpatrick08	& This Paper	& Geballe02     This Paper, Kirkpatrick08, Geballe02           \\
SDSS J085758.45+570851.4 & 				15.038$\pm$0.04	& 2.076$\pm$0.05		& L8	& Kirkpatrick08	& Burgasser10spex	& Geballe02	Burgasser10spex, Kirkpatrick08, Geballe02 \\
1043075+222523			 & 15.965$\pm$0.065 	& 1.974$\pm$0.077	& L8		& Cruz07	& This Paper	& Cruz07            This Paper, Cruz07, Cruz07                     \\
16322911+1904407	 & 					15.867$\pm$0.07	& 1.864$\pm$0.085		& L8	& K99	& This Paper	&                                               \\
\enddata

\tablecomments{These spectra were excluded from the templates are shown in Figure~\ref{fig:field_excluded}.}

\tablenotetext{a}{References pertain to the the spectral data used for spectral type and the object's discovery, if not already listed, and are not exhaustive.} 
\tablenotetext{b}{Spectral type based on analysis of spectrum from \cite{Scholz09}.}

\end{deluxetable}


%%%%%%%%%%%%%%%%%%%%%%%%%%%%%%%%
% LOW GRAVITY TEMPLATES    %
%%%%%%%%%%%%%%%%%%%%%%%%%%%%%%%%
\clearpage
\begin{figure}
  \epsscale{0.9}
  \plotone{L0strip_b.pdf}
  \plotone{L1strip_b.pdf}
  \caption{Spectral average templates for $\beta$-type low-gravity objects. The average is shown as a solid black line. At each pixel, the 1$\sigma$ variance is shown as dark gray shading and range of flux values are indicated by light gray. Objects included in the templates are listed on the left in order of increasing $J-K_S$ color. The most prominent atomic and molecular features are indicated. \added{The templates are available as Data behind the Figure.}}
  \label{fig:beta_templates}
\end{figure}

\begin{figure}
  \epsscale{0.9}
  \plotone{L0strip_g.pdf}
  \plotone{L1strip_g.pdf}
  \plotone{L2strip_g.pdf}
  \caption{Spectral average templates for $\gamma$-type very low-gravity objects. The average is shown as a solid black line. At each pixel, the 1$\sigma$ variance is shown as dark gray shading and range of flux values are indicated by light gray. Objects included in the templates are listed on the left in order of increasing $J-K_S$ color. The most prominent atomic and molecular features are indicated. \added{The templates are available as Data behind the Figure.}}
  \label{fig:gamma_templates}
\end{figure}

\begin{figure}
  \figurenum{9.2}
  \plotone{L3strip_g.pdf}
  \plotone{L4strip_g.pdf}
  \caption{Low-gravity $\gamma$ templates continued.}
\end{figure}

\begin{figure}
  \epsscale{0.9}
  \plotone{L0strip_b_excluded.pdf}
  \plotone{L0strip_g_excluded.pdf}
  \caption{Spectra excluded from the low-gravity $\beta$ and $\gamma$ templates (colors). The spectral average template is shown as a solid black line. Objects excluded from the templates are listed on the left in order of increasing $J-K_S$ color. The most prominent atomic and molecular features are indicated.}
  \label{fig:lowg_excluded}
\end{figure}

% included in LOW GRAVITY template
\clearpage
%!TEX root = /Users/kelle/Dropbox/Pubs IN PROGRESS/Paper XII NIR DATA/Cruz-Nunez/NIRLdwarfs.tex
\begin{deluxetable}{llllllll}
\floattable
\rotate
\tabletypesize{\tiny}
\tablewidth{0pt}
\tablecolumns{8}
\tablecaption{Low-Gravity L Dwarfs Included in the Low-Gravity Template Sample\label{tab:lowg_template}}
\tablehead{  \colhead{} & \colhead{} & \multicolumn{2}{c}{2MASS} \\
\cline{3-4} \\
\colhead{2MASS} & \colhead{Other} & \colhead{} & \colhead{} & \colhead{Optical} & \colhead{Sp. Type} & \colhead{SpeX Prism}  & \colhead{Other} \\
\colhead{Designation} & \colhead{Name} & \colhead{$J$} &  \colhead{$J-K_S$}  &  \colhead{Sp. Type} &
\colhead{Ref.} & \colhead{Ref.} & \colhead{Refs.\tablenotemark{a}}
}
\startdata
00325584$-$4405058 &    EROS-MP J0032$-$4405			 & 14.776$\pm$0.035	& 1.507$\pm$0.05	& L0$\gamma$	& \cite{Cruz09_lowg}         & This Paper         & \cite{EROSCollaboration:1999uj}		  \\
01415823$-$4633574	&		 & 14.832$\pm$0.043	& 1.735$\pm$0.054	& L0$\gamma$	& \cite{Cruz09_lowg}         & \cite{Kirkpatrick06}  & \cite{Kirkpatrick06} \\
02103857$-$3015313	&		 & 15.066$\pm$0.048	& 1.566$\pm$0.064	& L0$\gamma$	& This Paper         & This Paper         & \cite{Gagne:2015to}        \\
0241115$-$032658		&	 & 15.799$\pm$0.065	& 1.764$\pm$0.082	& L0$\gamma$	& \cite{Cruz09_lowg}         & This Paper         & \cite{Cruz07}        \\
03231002$-$4631237	&		& 15.389$\pm$0.07	& 1.687$\pm$0.086	& L0$\gamma$	& \cite{Cruz09_lowg}		& This Paper	& \cite{Reid08}  \\
1711135+232633		&	 & 14.499$\pm$0.026	& 1.443$\pm$0.037	& L0$\gamma$	& This Paper         & This Paper         & \cite{Cruz07}        \\
2213449$-$213607		&	 & 15.376$\pm$0.035	& 1.620$\pm$0.052	& L0$\gamma$	& \cite{Cruz09_lowg}         & This Paper          & \cite{Cruz07}        \\
23153135+0617146	&		 & 15.861$\pm$0.083	& 1.796$\pm$0.105	& L0$\gamma$	& This Paper         & This Paper         &        \\
\hline
03572695$-$4417305 & DENIS J0357$-$4417	&	  14.367$\pm$0.032	& 1.46 $\pm$0.042	& L0$\beta$	& \cite{Cruz09_lowg}  & This Paper       & \cite{Bouy03,Kirkpatrick08}        \\
11544223$-$3400390 & DENIS-P J115442.2$-$340039		 & 14.195$\pm$0.033	& 1.344$\pm$0.047	& L0$\beta$	& This Paper  & This Paper       & \cite{Bouy03,Kirkpatrick08}        \\
15525906+2948485	&		 & 13.478$\pm$0.026	& 1.456$\pm$0.038	& L0$\beta$	& \cite{Cruz09_lowg}        & This Paper       & \cite{Wilson03,Reid08}      \\
\hline
00452143+1634446	&		 & 13.059$\pm$0.022	& 1.693$\pm$0.03	& L1.5$\gamma$	& \cite{Reid08}        & This Paper         & \cite{Wilson03}      \\
0117474-340325		&	 & 15.178$\pm$0.036	& 1.689$\pm$0.052	& L1$\gamma$	& This Paper         & This Paper         & \cite{Cruz03}        \\
0518461-275645		&	 & 15.262$\pm$0.043	& 1.647$\pm$0.059	& L1$\gamma$	& This Paper         & This Paper         & \cite{Cruz07}        \\
\hline
02271036-1624479 & SIPS J0227-1624	& 13.573	$\pm$ 0.023 & 1.43	$\pm$ 0.038	& L1$\beta$	& \cite{Reid08}		& This Paper	& \cite{Deacon05} \\
10224821+5825453	&		 & 13.499$\pm$0.026	& 1.339$\pm$0.036	& L1$\beta$	& \cite{Cruz09_lowg}         & This Paper       & \cite{Reid08}       \\
\hline
00550564+0134365	&		 & 16.436$\pm$0.114	& 1.998$\pm$0.133	& L2$\gamma$	& This Paper         & This Paper         &         \\
0536199-192039		&	 & 15.768$\pm$0.074	& 1.914$\pm$0.097	& L2$\gamma$	& This Paper         & This Paper         & \cite{Cruz07}        \\
\hline
%1004207+502300 G 196-3B		 & 14.831$\pm$0.047	& 2.053$\pm$0.058	& L3$\beta$	& Cruz09         & This Paper       &               \\
15515237+0941148	&		 & 16.319$\pm$0.111	& 2.009$\pm$0.125	& L3.5$\gamma$	& This Paper         & This Paper         & \cite{Reid08}       \\
1726000+153819		&	 & 15.669$\pm$0.065	& 2.01 $\pm$0.082	& L3.5$\gamma$	& This Paper         & This Paper         & \cite{K00}           \\
22081363+2921215	&		 & 15.797$\pm$0.085	& 1.649$\pm$0.112	& L3$\gamma$	& \cite{Cruz09_lowg}          & This Paper       & \cite{K00}           \\
\hline
05012406-0010452	&		 & 14.982$\pm$0.038	& 2.019$\pm$0.052	& L4$\gamma$	& \cite{Cruz09_lowg}          & This Paper          & \cite{Reid08}       \\
15382417-1953116	&		 & 15.934$\pm$0.064	& 1.93 $\pm$0.081	& L4$\gamma$	& This Paper         & This Paper         &         \\
1615425+495321		&	 & 16.789$\pm$0.138	& 2.483$\pm$0.155	& L4$\gamma$	& This Paper         & This Paper         & \cite{Cruz07}        \\
2206449-421720		&	 & 15.555$\pm$0.066	& 1.946$\pm$0.086	& L4$\gamma$	& \cite{K00}            & This Paper         & \cite{K00}           \\
SDSS 2249+0044	&	 & 16.587$\pm$0.125	& 2.229$\pm$0.143	& L4$\gamma$	& This Paper & \cite{Allers:2010cg}       & \cite{Geballe02,Kirkpatrick08}   \\
\enddata

\tablecomments{Templates generated using spectra of these objects are shown in Figures~\ref{fig:beta_templates} and \ref{fig:gamma_templates}.}

\tablenotetext{a}{References listed pertain to discovery and spectral data and are not exhaustive.}

\end{deluxetable}



% EXCLUDED from low gravity template
\clearpage
%!TEX root = /Users/kelle/Dropbox/Pubs IN PROGRESS/Paper XII NIR DATA/Cruz-Nunez/NIRLdwarfs.tex
\begin{deluxetable}{lllllllll}
\tabletypesize{\tiny}
%\rotate
\tablewidth{0pt}
\tablecolumns{7}
\tablecaption{Low-Gravity L Dwarfs Excluded from the Low-Gravity Template Sample\label{tab:lowg_excluded}}
\tablehead{  \colhead{} & \colhead{} &  \multicolumn{2}{c}{2MASS All-Sky PSC} \\
\cline{3-4} \colhead{} & \colhead{Other} & \colhead{}  & \colhead{} & \colhead{Optical} & \colhead{Sp. Type} & \colhead{SpeX Prism}  & \colhead{Other} \\
\colhead{Object} & \colhead{Name} & \colhead{$J$} &  \colhead{$J-K_S$}  &  \colhead{Sp. Type} & \colhead{Ref.} & \colhead{Ref.} & \colhead{Refs.\tablenotemark{a}}
}
\startdata
03483060+2244500 & Roque 25	& 17.79		$\pm$ 0.25	& 1.44	$\pm$ 0.354	& L0$\beta$	& \cite{Kirkpatrick08}	& This Paper	& \cite{Martin98}         \\
04062677-3812102 &			& 16.768	$\pm$ 0.126 & 1.655	$\pm$ 0.172	& L0$\gamma$	& \cite{Kirkpatrick10}	& \cite{Kirkpatrick10}	& \cite{Kirkpatrick10} \\
\enddata
\tablecomments{Low gravity L dwarfs not included in templates and are shown in Figure~\ref{fig:lowg_excluded}.}
\tablenotetext{a}{References listed are not exhaustive and pertain to discovery and spectral data.}
\end{deluxetable}



\section{Analysis: Field-Gravity L Dwarfs}
\label{sec:fieldg}

\subsection{The Field-Gravity Template Subsample}
\label{sec:templates_normal}

Our goal is to use the optical-NIR sample to generate spectral average templates that are representative of the field population of L dwarfs and that reflect the intrinsic spread in spectral morphology due to the range of properties of the thin disk population.
However, the statistics and biases of the optical-NIR sample are different at each spectral type and the resulting templates likely do not represent the same segments of the bulk population.
Below, we examine the spectral type and $J-K_S$ color distributions of the field-gravity objects in the optical-NIR sample and discuss features which could be attributed to observational biases in the sample versus intrinsic properties of the population and how these might impact the resulting templates.

The spectral type distribution of the optical-NIR sample is shown in Figure~\ref{fig:spthist}.
The field-gravity objects that are included in the templates are shown as black bars while the excluded objects are shown in gray.
There are at least eight objects included in each template.
At the earlier types (L0--L3), more objects were considered and a higher fraction were rejected compared to the later types (L4, L6--L8).
% more earlier type objects
More earlier-type objects are present in the optical-NIR sample for two primary reasons: First, the number densities of earlier-type objects are intrinsically higher than later-type objects \citep{Cruz03,Reid08,DayJones:2013hm}.
Second, earlier-type objects are more luminous than later-type objects and thus there are more known and there are more data available for them.
% higher fraction rejected at earlier types
The later- and earlier-type spectra display a similar degree of homogeneity (as reflected in the templates by the flux range and 1$\sigma$ dispersion at each pixel), even though the former include a higher fraction of objects in the templates.
As more objects are added to the optical-NIR sample, it will be interesting, especially at later types, to see how these distributions change.

%No L9 type
The analysis presented here is anchored in the optical and thus we do not have an L9 template.
While the optical system of \cite{K99} only goes to L8, \cite{Geballe02} introduced an L9 type, mostly to maintain smoothly varying infrared spectral indices.
\cite{Geballe02} define the start of the T dwarf sequence as the first appearance of the 1.6$\micron$ methane absorption feature in the \emph{H}~band.
All L8-type objects have a 2.2$\micron$ methane absorption feature in the \emph{K}~band.
\added{The sample includes several objects classified as an L8 in the optical and L9 in the NIR (e.g., 0255-47, 0328+23, 0830+48).}
While we do see a larger variety of NIR spectral morphologies among the optically-classified L8~objects, we do not see any trends in the spectral features which justify a spectral class later than L8.

In Figure~\ref{fig:JK_colors}, we show 2MASS $J-K_S$ color versus optical spectral type for the optical-NIR sample of L~dwarfs.
We display the $J-K_S$ color distribution of the objects included in each template by using a box plot representation. The height of the box indicates the interquartile range (IQR), that is, the range of the middle half of the data, while the whiskers show the extent of the lower and upper quartiles. The median is indicated by a horizontal bar inside the box. For bins with five or fewer objects, only whiskers are shown indicating the range of the distribution. Excluded objects are shown individually as gray x marks.
The number of objects included in each template, the mean, median, minimum, maximum, and range of $J-K_S$ colors are listed in Table~\ref{tab:sample_prop}.
The properties of these distributions reveal several things of note:
\begin{itemize}
\item Some field-gravity objects excluded from the templates have near-average colors. This implies that field-gravity objects with peculiar spectra may not have extreme colors.
\item Some field-gravity objects with colors in the red/blue tails of the color distribution are included in the templates. This implies field-gravity objects with extreme color may not display any spectral peculiarities besides spectral slope.
\item The IQR for the field-gravity objects included in the templates is much smaller for L0--L3 ($IQR\le0.13~mags$) compared to L4--L6 ($IQR\ge0.25~mags$).
This difference cannot entirely be attributed to small number statistics at later types because there are nearly the same number of L5~objects considered ($N=15,IQR=0.25~mags$) as L3~objects ($N=14,IQR=0.07~mags$).
We note that the early-type population is expected to be dominated by the lowest-mass stars (70--80~$M_{Jup}$) while the later-type population is expected to be entirely substellar \citep{Burrows01}.
It is possible that the larger color spread at later types is due to the larger range of underlying masses, ages, and gravities expected in the brown dwarf population compared to the more homogeneous population of the lowest-mass stars dominating the earlier spectral types.
It is also possible that at the cooler temperatures/later types, the role of small dust grains in the photospheres becomes more important and small changes in those dust properties have a larger impact on the emergent spectra \citep{Burgasser08_blue,Marocco:2014kr,Hiranaka:2016jm}.
\end{itemize}

%compare to Faherty13 J-K color distribution
In order to put the field-gravity optical-NIR sample in the context of the entire known field L~dwarf population, we compare the $J-K_S$ color distributions of the objects included in the templates with those of \citet{Faherty13_0355}.
They compiled a sample of photometrically-defined field L~dwarfs with photometric uncertainties smaller than 0.1~mag.
The Faherty et al. averages are plotted next to those of our field-gravity template sample in Figure~\ref{fig:JK_colors_F13}.
The range of $J-K_S$ colors spanned by each sample are indicated by vertical lines at each spectral type, and the number of L~dwarfs used to calculate each average is indicated above each color range.
The color range of our restrictive, spectroscopically selected sample is not as large as the more inclusive Faherty~et~al. photometrically-selected sample.
The average colors of our field-gravity template sample agree well with the average colors of Faherty~et~al.'s less-restrictive field sample from L0--L3.
However, there does indeed appear to be a bias in the template sample that favors the red side of the color distribution.
In particular, the L1~template sample contains more objects redder than the average and the averages of the L4 and L5~template objects are redder than the Faherty~et~al. field average.
The average colors and range of the L6--L8 objects included in the templates agree as well as could be expected with the Faherty~et~al. field averages given the small numbers being considered ($N\le10$).
This general agreement between the color distributions of the template sample and the Faherty~et~al. sample and the striking homogeneity of the spectra at each spectral type gives us confidence in the templates' representation of the field L~dwarf population's NIR spectral morphology, especially at types L0--L4.

\begin{figure}
  \epsscale{0.9}
    \plotone{JK_JF.pdf}
    \caption{Average 2MASS $J-K_S$ color versus spectral type for two samples of L dwarfs. The squares are the averages $J-K_S$ colors of the L dwarfs in our normal optical-NIR sample, and the circles are the average $J-K_S$ colors of the \citet{Faherty13_0355} compilation of field L dwarfs. The extent of the vertical lines indicates the range of $J-K_S$ colors spanned by each sample at each spectral type. The number of L dwarfs used to calculate the average $J-K_S$ color at each spectral type is indicated above the range. Both samples only include L dwarfs that have photometric uncertainties smaller than 0.1~mag. For spectral types L0--L5, the mean colors of our normal optical-NIR sample are consistent with those of Faherty's larger sample. The difference in color ranges spanned by the two samples is because of the much stricter spectroscopic selection criteria imposed for the optical-NIR sample compared to the photometrically-selected Faherty et al. sample.}
  \label{fig:JK_colors_F13}
\end{figure}

\clearpage
%!TEX root = /Users/kelle/Dropbox/Pubs IN PROGRESS/Paper XII NIR DATA/Cruz-Nunez/NIRLdwarfs_Normal.tex
\begin{deluxetable}{cllllllr}
\tablewidth{0pt}
%\tabletypesize{\footnotesize}
\tabletypesize{\small}
%\tabletypesize{\scriptsize}
%\tabletypesize{\tiny}
%\rotate
\tablewidth{0pt}
\tablecolumns{7}
\tablecaption{Properties of the Field-Gravity Optical-NIR Template Sample\label{tab:sample_prop}}
\tablehead{
\colhead{} &  \multicolumn{6}{c}{$J-K_S$} & \colhead{} \\
\cline{2-7} \\
\colhead{Optical} &  \colhead{} &  \colhead{1st/2nd} & \colhead{} & \colhead{3rd/4th} &  \colhead{} &  \colhead{} & \colhead{No. of}\\
\colhead{Spectral Type} &  \colhead{Min} & \colhead{Quartile} & \colhead{Median} & \colhead{Mean} & \colhead{Quartile}  & \colhead{Max}  & \colhead{Objects}
}
\startdata
L0                  & 1.06                  & 1.24 & 1.29                  & 1.27 & 1.30 & 1.41                  & 14 \\
L1                  & 1.17                  & 1.31 & 1.36                  & 1.36 & 1.39 & 1.58                  & 18 \\
L2                  & 1.32                  & 1.38 & 1.47                  & 1.47 & 1.51 & 1.67                  & 13 \\
L3                  & 1.41                  & 1.58 & 1.61                  & 1.63 & 1.65 & 1.85                  & 14 \\
L4                  & 1.43                  & 1.72 & 1.90                  & 1.83 & 1.99 & 2.10                  & 8 \\
L5                  & 1.57                  & 1.73 & 1.89                  & 1.85 & 1.98 & 2.03                  & 15 \\
L6                  & 1.54                  & 1.59 & 1.66                  & 1.77 & 1.96 & 2.14                  & 9 \\
L7\tablenotemark{*} & 1.85\tablenotemark{*} & 2.01 & 2.07\tablenotemark{*} & 1.02 & 2.07 & 2.08\tablenotemark{*} & 4\tablenotemark{*} \\
L8                  & 1.47                  & 1.69 & 1.72                  & 1.70 & 1.77 & 1.80                  & 6 \\
\enddata

\tablecomments{Values describing the  $J-K_S$ color distribution of the objects included in the low gravity spectral templates. The minimum, maximum, median, mean, and boundaries of the quartiles are listed in addition to the number of objects included in each template. These values are plotted in Figure~\ref{fig:JK_colors}.}

\tablenotetext{*}{There are very few L7 type objects in the sample, their average color is not consistent with the \cite{Faherty13_0355} sample (Figure~\ref{fig:JK_colors_F13}) and these should not be considered representative of the entire population.}
\end{deluxetable}



\subsection{Isolating Temperature: The Field L Dwarf Spectral Sequence}
\label{sec:temp}
We compare the field L dwarf \emph{zJ}, \emph{H}, and $K_S$ low-resolution spectral templates described in the previous subsection in Figure~\ref{fig:spec_sequence}.
The prominent atomic (\ion{K}{1}, \ion{Na}{1}) and molecular (FeH, H$_2$O) absorption spectral features from L0 to L5 form a clear sequence in all three bands (top panel).
\added{(See \citet{Cushing05} for more precise identifications of these spectral features.)}
These same spectral features in the later-type objects (L6--L8), however, saturate (H$_2$O) or disappear (FeH, \ion{K}{1}) and do not follow the same trends as the earlier-type objects. Therefore, we plot them separately (bottom panel).

%%%%%%%%%%%%%%%%%%
% FIELD SEQUENCE %
%%%%%%%%%%%%%%%%%%

\begin{figure}
    \epsscale{1}
    \plotone{sequence0_f.pdf}
    \plotone{sequence1_f.pdf}
    \caption{Spectral band templates overplotted, L0--L5 on the top panels, and L6--L8 on the bottom panels.
    The L0--L5 spectra form an organized sequence in the \emph{zJ}, \emph{H}, and \emph{K} bands, whereas the L6--L8 spectra do not follow the same trend as the earlier types and are, therefore, plotted separately.}
    \label{fig:spec_sequence}
\end{figure}

Below, we describe how the spectral features in the low-resolution L0--L4 template sequence change, as shown in the top panels of Figure~\ref{fig:spec_sequence}. (The L5--L8 spectra do not follow this sequence.)
While these trends have been known since the early studies of L dwarfs in the NIR \citep{Testi01,Reid01_NIR,Geballe02}, we would like to emphasize their prominence in the optically-anchored templates.
Since these templates are anchored to the optical spectral types, which has been shown to be primarily sensitive to temperature, we deduce that these NIR spectral changes are also primarily modulated by temperature.
With later type/cooler temperature,
\begin{itemize}
\item the spectral slope in \emph{zJ} band gets redder\added{,}
\item FeH absorption (1.0, 1.6~$\micron$) increases,
\item VO absorption (1.06, 1.18~$\micron$) decreases,
\item H$_2$O absorption (1.4, 1.9~$\micron$) increases,
\item the slope of the blue side of \emph{H} band and the red side of \emph{K} band increases (e.g., water absorption is less broad, but deeper),
\item the \emph{K} band shape becomes more sharply peaked,
\item the presence of \ion{Na}{1} (1.14, 2.2~$\micron$) becomes less common, and
\item the strength of CO absorption does not significantly change.
\end{itemize}

The transition to L6 in the NIR spectra is marked most notably by the reversal of the strengthening of FeH absorption (1.0, 1.6~$\micron$).
At L6, the FeH absorption begins to weaken and is not present by L7.
The disappearance of FeH results in a smoother \emph{zJ} and \emph{H}~band spectral shape in the L6--L8 objects.
In addition, the onset of water at 1.1~$\micron$ contributes to the smooth \emph{zJ}-band shape in the L7 and L8 objects.
Following the trend seen in L0--L5, the depth of the water bands continue to increase and \emph{narrow}---due to the fewer high energy rotational transitions at lower temperatures---resulting in steeper slopes on the blue sides of the \emph{H}~and \emph{K}~bands.
In these later types, the \emph{K}-band slope towards the CO is much steeper.
Finally, L8 is marked by the onset of methane absorption in the \emph{K}-band.

\section{Analysis: Low-Gravity L Dwarfs}
\label{sec:lowg}

\subsection{The Low-Gravity Template Subsample}
\label{sec:templates_lowg}

Following the method described in \S~\ref{sec:templates}, we used the optical-NIR sample to generate optically anchored spectral average templates for L0$\gamma$--L4$\gamma$, L0$\beta$, and L1$\beta$.
($\gamma$ and $\beta$ spectral type suffixes are used to indicate very low-gravity and intermediate-gravity spectral features, respectively.)
Of these \optLowG~low-gravity objects, there are \optGamma~L0--L4 objects with a $\gamma$ gravity classification and \optBeta~L0--L1 $\beta$ type objects.
This low-gravity sample is composed of objects from \citet{Cruz07,Kirkpatrick08,Cruz09_lowg} and ten additional objects with new data presented in \S\S~\ref{sec:obs_new_opt} and~\ref{sec:obs_new_nir}.
Many other known low-gravity objects are not included in this analysis because they lack SpeX Prism data as a result of their southerly declinations.

The spectral type distribution of the low-gravity objects is shown in Figure~\ref{fig:spthist} as red and orange bars.
This sample of low-gravity objects is small and incomplete\added{,} but the striking homogeneity of their NIR spectral morphologies amongst objects of the same optical spectral type provided strong motivation for \replaced{making}{constructing} templates \replaced{at}{for} some spectral types.

\replaced
{The low-gravity templates are shown in Figure Sets~\ref{fig:beta_templates}--\ref{fig:gamma_templates} and the targets included in them are listed in Table~\ref{tab:lowg_template}.}
{Low-gravity templates were produced following the same procedure for our field gravity templates.
These are shown in Figure Sets~\ref{fig:beta_templates}--\ref{fig:gamma_templates}, with the targets included in them are listed in Table~\ref{tab:lowg_template}.}
The spectra rejected from the low-gravity templates are shown in Figure Set~\ref{fig:lowg_excluded} and listed in Table~\ref{tab:lowg_excluded}.

\replaced{Of the~\optLowG, only two were excluded from the templates and all primarily due to the low signal-to-noise ratio of their NIR spectra, rather than because of their inconsistent spectral morphology}{Of the~\optLowG~low-gravity spectra available, only two were excluded from the templates. They were excluded primarily due to the low signal-to-noise ratio of their NIR spectra} (Table~\ref{tab:lowg_excluded}, Figure Set~\ref{fig:lowg_excluded}).

In Figure~\ref{fig:JK_colors}, we show the 2MASS $J-K_S$ color distribution versus spectral type for the field (black and gray) and low-gravity objects (red and orange) included in the optical-NIR sample of L~dwarfs.
For the low-gravity L0 bin, we use a box plot representation for the color distribution, while for the low-gravity L1--L4, where there are fewer than five objects in each bin, we only show the whiskers indicating the range of the distribution.
Not surprisingly, the $J-K_S$ color distribution of the low-gravity objects is significantly redder than the field-gravity objects.
However, the two distributions also have notable overlap: there are objects with quite red colors for their spectral type that do not have any recognized signatures of low gravity in their optical or low-resolution NIR spectra.
This further supports the notion that most low-gravity objects are red, but not all red objects are low gravity.
While diversity of cloud properties is often pointed to as the primary suspect for the red colors of low-gravity objects, a significant amount of further theoretical and observational work is needed to pinpoint the nature of the physical properties modulating the $J-K_S$ colors of L dwarfs.

\clearpage
%!TEX root = /Users/kelle/Dropbox/Pubs IN PROGRESS/Paper XII NIR DATA/Cruz-Nunez/NIRLdwarfs.tex
\begin{deluxetable}{cllllllr}
\tablewidth{0pt}
%\tabletypesize{\footnotesize}
\tabletypesize{\small}
%\tabletypesize{\scriptsize}
%\tabletypesize{\tiny}
%\rotate
\tablewidth{0pt}
\tablecolumns{7}
\tablecaption{Properties of the Low-Gravity Optical-NIR Template Sample\label{tab:lg_sample_prop}}
\tablehead{
\colhead{} &  \multicolumn{6}{c}{$J-K_S$} & \colhead{} \\
\cline{2-7} \\
\colhead{Optical} &  \colhead{} &  \colhead{1st/2nd} & \colhead{} & \colhead{} & \colhead{3rd/4th} &   \colhead{} & \colhead{No. of}\\
\colhead{Spectral Type} &  \colhead{Min} & \colhead{Quartile} & \colhead{Median} & \colhead{Mean} & \colhead{Quartile}  & \colhead{Max} &  \colhead{Objects}
}
\startdata
\cutinhead{$\gamma$-type objects}
L0$\gamma$            & 1.44 & 1.55 & 1.64 & 1.64 & 1.74 & 1.80 & 8 \\
L1$\gamma$            & 1.65 & 1.67 & 1.69 & 1.68 & 1.69 & 1.69 & 3 \\
L2$\gamma$            & 1.91 & \nodata & 1.96 & 1.96 & \nodata & 2.00 & 2  \\
L3$\gamma$            & 1.65 & \nodata & 2.01 & 1.89 & \nodata & 2.01 & 3  \\
L4$\gamma$            & 1.93 & 1.95 & 2.02 & 2.12 & 2.23 & 2.48 & 5  \\
\cutinhead{$\beta$-type objects}
L0$\beta$             & 1.34 & \nodata & 1.46 & 1.42 & \nodata & 1.46 & 3 \\
L1$\beta$             & 1.34 & \nodata & 1.38 & 1.38 & \nodata & 1.43 & 2  \\
\cutinhead{Combined $\beta$- and $\gamma$-type objects}
L1$\gamma$ \& L1$\beta$ & 1.34 & \nodata & 1.65 & 1.56 & \nodata &  1.69 & 5  \\
L0$\gamma$ \& L0$\beta$ & 1.34 & 1.46 & 1.57 & 1.58 & 1.71 &  1.80 & 11 \\
\enddata
\tablecomments{Values describing the  $J-K_S$ color distribution of the objects included in the low gravity spectral templates.
The minimum, maximum, median, mean, and boundaries of the quartiles are listed in addition to the number of objects included in each template.
Only the median, mean, minimum, and maximum are shown for templates with five objects or less.
These values are plotted in Figure~\ref{fig:JK_colors}. For L0 and L1, where we have both $\beta$ and $\gamma$ low-gravity objects, the statistics are listed for the subgroups individually (8 and 3 objects for L0, 3 and 2 objects for L1) and combined (11 total for L0 and 5 total for L1). The combined low-gravity stats are shown in Figure~\ref{fig:JK_colors}.}
\end{deluxetable}


\subsection{The low-gravity spectral sequence}

The optically anchored low-gravity L0--L4 $\gamma$ dwarf \emph{zJ}, \emph{H}, and \emph{K} low-resolution spectral templates are shown overplotted on top of each other in Figure~\ref{fig:lg_sequence}.
Many of the spectral trends in the low-gravity sequence are similar to those in the field L0--L4 sequence but smaller in amplitude. And as with the field-gravity objects, we assume these changes are primarily temperature dependent. The recognized gravity-sensitive features also change through the L0--L4 $\gamma$ sequence, and again, we attribute these changes to cooler temperatures.
In low-gravity spectra, with later type/cooler temperature,
\begin{itemize}
\item the spectral slope in \emph{zJ} band gets redder
\item FeH absorption (1.0~$\micron$) remains essentially constant,
\item VO absorption at 1.06~$\micron$ decreases and disappears by L4,
\item VO absorption at 1.18~$\micron$ remains strong but gradually weakens,
\item Depth of the \ion{K}{1}, \ion{Na}{1} doublets at 1.14, 1.17, 1.25~$\micron$ decreases,
\item \emph{H} band slopes get steeper and shape becomes more pointed,
\item H$_2$O absorption (1.4, 1.9~$\micron$) increases, and
\item \emph{K} band slopes get steeper.
\end{itemize}

%%%%%%%%%%%%%%%%%%
% GAMMA SEQUENCE %
%%%%%%%%%%%%%%%%%%

\begin{figure}
  \epsscale{1}
  \plotone{sequence0_g.pdf}
  \caption{The low-gravity $\gamma$ L dwarf near-infrared spectral sequence.}
  \label{fig:lg_sequence}
\end{figure}

The low-gravity $\gamma$ sequence does not display striking spectral morphology changes between types, especially compared to the field sequence shown in Figure~\ref{fig:spec_sequence}.
The most prominent features discriminating the types are the $J$~band slope and the strength of the 1.4$\micron$~H$_2$O absorption and the blue-side \emph{H}~band slope, both of which are subtle.
This dearth of discriminating spectral features contributes to the challenge of spectral typing low-gravity L dwarfs.
As discussed in later sections of this paper, it might be appropriate in many cases to report a range of valid spectral types for a given low-resolution spectrum.
As more objects are discovered and higher signal-to-noise data are obtained, the sequence may become better defined.
However, it is also possible that the spectral sequence of low-gravity L dwarfs is simply coarser than the field-gravity sequence at this spectral resolution.

Low-gravity $\gamma$ objects with spectral types later than L4$\gamma$ are not included in our analysis.
There are fewer objects at these late types and they exhibit a wide diversity of spectral morphologies.
Preliminary results on the low-gravity sequence past L4$\gamma$ is presented by \cite{Gagne:2015dc}.
For a future paper, we plan to undertake a careful analysis of these data with the goal of generating templates.

\subsection{Isolating gravity}

The differences between the NIR spectral morphology of low- and field-gravity spectra of L~dwarfs have been described before \citep{Lucas01,McGovern04,Kirkpatrick06,Allers:2013hk}. Briefly, the distinguishing characteristics of a low-gravity NIR spectrum compared to a field-gravity one at the same spectral type are:
\begin{itemize}
\item $J$ band slope is redder,
\item FeH absorption 1.0~$\micron$ is weaker,
\item VO absorption (1.06, 1.18~$\micron$) is stronger,
\item \ion{K}{1}, \ion{Na}{1} doublets at 1.14, 1.17, 1.25~$\micron$ are weaker,
\item H$_2$O absorption (1.4, 1.9~$\micron$) is stronger and slopes are steeper,
\item red side of the \emph{H} band slope is steeper resulting in triangular shaped \emph{H} band,
\item 1.6~$\micron$ FeH absorption is not present, and
\item blue side of the \emph{K} band slope is steeper.
\end{itemize}

One of the potential uses of the templates described in \S~\ref{sec:templates} is to reliably distinguish field-gravity L dwarfs from low-gravity and potentially young L dwarfs using only NIR spectra.
Below, we compare each low-gravity template to the suite of field-gravity templates and highlight the distinguishing features and the minimum recommended observations for a reliable gravity classification using low-resolution NIR spectra.

\begin{description}
\item[L0$\gamma$]{
In Figure~\ref{fig:L0lg-field}\added{,} the L0$\gamma$ spectral average template is compared to the field L dwarf templates.
The \emph{zJ} and \emph{H} bands of the L0$\gamma$ template are distinctive from all field L dwarf templates.
In particular, the \emph{zJ} band of the L0$\gamma$ type displays weaker FeH absorption at 0.99--1.007 $\mu$m and stronger VO absorption at 1.05--1.08 $\mu$m and triangular-shaped \emph{H} band compared to the field objects.
Aside from these notable differences, the \emph{zJ}-band shape is similar to the L1 and L2 field templates.
The L0$\gamma$ \emph{K}-band shape, on the other hand, is only subtly different from the L0 field template, with a slightly flatter and less round shape and a weaker CO absorption band head at 2.3~$\mu$m.
\emph{On the basis of these data, we conclude that in order to use low-resolution NIR data to credibly distinguish an L0$\gamma$ from a field object, a \emph{K}-band spectrum is insufficient and \emph{H}- and/or \emph{zJ}-band data are required.}
}

\begin{figure}[b]
    \epsscale{1}
    \plotone{templates-L0g.pdf}
    \caption{The L0$\gamma$ spectral average template (blue) compared to the field L dwarf templates and their 1$\sigma$ variance (black, gray). We conclude that in order to use low-resolution NIR spectra to credibly distinguish an L0$\gamma$ from a field object in the NIR, a \emph{K}-band spectrum is insufficient and \emph{H} and/or \emph{J}-band data are required.}
    \label{fig:L0lg-field}
\end{figure}
\clearpage

\item[L0$\beta$]{
In Figure~\ref{fig:L0b-field}\added{,} the L0$\beta$ spectral average template is compared to the field L dwarf templates.
The $J$-band and \emph{K}-band spectra of the L0$\beta$ template is similar to the L0, L1 and L2 field templates.
On the other hand, the \emph{H}-band of the L0$\beta$ template displays the hallmark triangular-shape and is distinctive from the field templates.
\emph{On the basis of these data, we conclude that in order to use low-resolution NIR data to credibly distinguish an L0$\beta$ from a field object in the NIR, \emph{zJ} and/or \emph{K}-band data are insufficient and \emph{H}-band data are required.}
}

\begin{figure}[b]
    \plotone{templates-L0b.pdf}
    \caption{The L0$\beta$ spectral average template (blue) compared to the field L dwarf templates and their 1$\sigma$ variance (black, gray). We conclude that in order to use NIR spectra to credibly distinguish an L0$\beta$ from a field object using low-resolution NIR data, \emph{zJ}-band and/or \emph{K}-band spectra is insufficient and \emph{H} data are required.}
    \label{fig:L0b-field}
\end{figure}
\clearpage

\item[L1$\gamma$]{
In Figure~\ref{fig:L1lg-field}\added{,} the L1$\gamma$ spectral average template is compared to the field L dwarf templates.
Like the L0$\gamma$, the L1$\gamma$ displays weaker FeH absorption, stronger VO absorption, and triangular-shaped \emph{H} band compared to the field L dwarf templates thus making the \emph{zJ} and \emph{H} bands of the L1$\gamma$ template distinctive.
Aside from these differences, the \emph{zJ}-band shape of the L1$\gamma$ template is most similar to the L2 and L3 field templates.
The L1$\gamma$ \emph{K}-band shape is only subtly different from the early-type (L0--L3) field templates and most closely resembles the L1 field template with just a weaker CO absorption band head at 2.3~$\mu$m.
\emph{On the basis of these data, we conclude that in order to use low-resolution NIR data to credibly distinguish an L1$\gamma$ from a field object, a \emph{K}-band spectrum is insufficient and \emph{H}- and/or \emph{zJ}-band data are required.}
}

\begin{figure}[b]
        \plotone{templates-L1g.pdf}
    \caption{The L1$\gamma$ spectral average template (light blue) to the field L dwarf templates and their 1$\sigma$ variance (black, gray). We conclude that in order to use low-resolution NIR spectra to credibly distinguish an L1$\gamma$ from a field object in the NIR, either a \emph{J}, \emph{H}, or \emph{K}-band spectrum is sufficient.}
    \label{fig:L1lg-field}
\end{figure}

\begin{figure}
  \figurenum{16.2}
  \plotone{templates2-L1g.pdf}
  \caption{L1$\gamma$ comparison cont.}
\end{figure}
\clearpage

\item[L1$\beta$]{
In Figure~\ref{fig:L1b-field}\added{,} the L1$\beta$ spectral average template is compared to the field L dwarf templates.
The L1$\beta$ template is indistinguishable from the field template.
\emph{On the basis of these data, we conclude that low-resolution NIR data cannot be used to credibly distinguish an L1$\beta$ from a field \textbf{L0--L2} object in the NIR.}}

\begin{figure}[b]
    \plotone{templates-L1b.pdf}
    \caption{The L1$\beta$ spectral average template (red) compared to the field L dwarf templates and their 1$\sigma$ variance (black, gray). The L1$\beta$ is indistinguishable from the L1 field template and thus we conclude that low-resolution NIR spectra cannot be used to credibly distinguish an L1$\gamma$ from a field object.}
    \label{fig:L1b-field}
\end{figure}
\clearpage

\item[L2$\gamma$]{
In Figure~\ref{fig:L2lg-field}\added{,} the L2$\gamma$ spectral average template is compared to all the other field L dwarf templates.
The L2$\gamma$ type is distinctive from field dwarfs in all three \emph{zJ}, \emph{H}, and \emph{K}-bands.
Even more pronounced than in the L0$\gamma$ and L1$\gamma$ templates, the L2$\gamma$ \emph{zJ}-band spectrum has weaker FeH and stronger VO molecular absorption compared to field L dwarfs.
In addition, there is also a hint of weaker \ion{K}{1} absorption at 1.25~$\mu$m. Aside from these differences in particular absorption features, the overall shape of the L2$\gamma$ \emph{zJ} band is most similar to the L5 field template (not shown).
The L2$\gamma$ \emph{H} band most closely resembles that of the L7 field template but with the hallmark  triangular-shape (not shown).
Finally, the L2$\gamma$ \emph{K}-band shape resembles that of early-type (L0--L2) field objects, except with a distinctive redder overall slope and significantly weaker CO absorption band head at 2.3~$\mu$m.
\emph{On the basis of these data, we conclude that in order to use low-resolution NIR data to credibly distinguish an L2$\gamma$ from a field object in the NIR, either a \emph{zJ}, \emph{H}, or \emph{K}-band spectrum is sufficient.}
}

\begin{figure}[b]
    \plotone{templates-L2g.pdf}
    \caption{The L2$\gamma$ spectral average template (green) to the field L dwarf templates and their 1$\sigma$ variance (black, gray).
    We assert that in order to use low-resolution NIR spectra to credibly distinguish an L2$\gamma$ from a field object in the NIR, either a \emph{J}, \emph{H}, or \emph{K}-band spectrum is sufficient.}
    \label{fig:L2lg-field}
\end{figure}

\begin{figure}
  \figurenum{18.2}
  \plotone{templates2-L2g.pdf}
  \caption{L2$\gamma$ comparison cont.}
\end{figure}
\clearpage

\item[L3$\gamma$]{
In Figure~\ref{fig:L3lg-field}\added{,} the L3$\gamma$ spectral average template is compared to all the other field L dwarf templates.
The L3$\gamma$ type is distinctive from field dwarfs in all three \emph{zJ}, \emph{H}, and \emph{K}-bands.
the L3$\gamma$ \emph{zJ}-band spectrum most closely resembles the L4 and L5 field templates but with weaker FeH and stronger VO molecular absorption.
The L3$\gamma$ \emph{H}-band most closely resembles that of the L7 field template but with the hallmark triangular-shape.
The L3$\gamma$ \emph{K}-band shape is distinctive from all of the field templates but most closely resembles that of early-type (L1--L3) field objects.
\emph{On the basis of these data, we conclude that in order to use low-resolution NIR data to credibly distinguish an L3$\gamma$ from a field object in the NIR, either a \emph{zJ}, \emph{H}, or \emph{K}-band spectrum is sufficient.}
}

\begin{figure}[b]
  \plotone{templates-L3g.pdf}
  \caption{The L3$\gamma$ spectral average template (orange) to the field L dwarf templates and their 1$\sigma$ variance (black, gray).
  We assert that in order to use low-resolution NIR spectra to credibly distinguish an L3$\gamma$ from a field object, a \emph{K}-band spectrum is insufficient and \emph{H}- and/or \emph{J}-band data are required.}
  \label{fig:L3lg-field}
\end{figure}

\begin{figure}
    \figurenum{19.2}
     \plotone{templates2-L3g.pdf}
     \caption{L3$\gamma$ comparison cont.}
\end{figure}
\clearpage

\item[L4$\gamma$]{
In Figure~\ref{fig:L4lg-field}, the L4$\gamma$ spectral average template is compared to all the other field L dwarf templates.
The \emph{zJ} band of the L4$\gamma$ template looks very similar to the field L4 and L5 templates but with weaker 1$\micron$ FeH absorption and a redder overall slope. (Unlike in the earlier-type $\gamma$ templates, there is no distinctive VO absorption.)
Just like the earlier-type $\gamma$ templates, the triangular-shaped \emph{H} band is quite distinctive from the \emph{H} band shape of the field objects. And similar to the L2 and L3$\gamma$, the \emph{H} band of the L4$\gamma$ template most resembles that of the L7 field template.
Finally, the L4$\gamma$ \emph{K}-band shape closely resembles that of the earlier-type (L1--L3) field templates, but with weaker CO absorption at 2.3~$\mu$m.
\emph{On the basis of these data, we conclude that in order to use low-resolution NIR data to credibly distinguish an L4$\gamma$ from a field object in the NIR, \emph{zJ} and/or \emph{K}-band data are insufficient and \emph{H}-band data are required.}
}
\end{description}

\begin{figure}[b]
    \plotone{templates-L4g.pdf}
    \caption{The L4$\gamma$ spectral average template (dark red) to the field L dwarf templates.
    We assert that in order to use low-resolution NIR spectra to credibly distinguish an L4$\gamma$ from a field object, a \emph{K}-band spectrum is insufficient and H and/or \emph{J}-band data are required.}
    \label{fig:L4lg-field}
\end{figure}

\begin{figure}
     \figurenum{20.2}
     \plotone{templates2-L4g.pdf}
     \caption{L4$\gamma$ comparison cont.}
\end{figure}
\clearpage

\section{Near-Infrared Spectral Classification of L Dwarfs}
\label{sec:classification}

As discussed previously, several NIR spectral classification methods for L dwarfs have been proposed that rely on spectral indices and/or do not use the entire NIR spectral regime.
While quantitative, index based schemes are useful, a scheme is also needed which uses spectral standards and the entirety of the spectrum to estimate a type, as in the MK system \citep{Morgan:1984wy,Cushing:2014un}.
K10 identified spectral standards (hereafter, K10 standards) that provide a \emph{zJ}-band sequence and a scheme which uses the \emph{H} and \emph{K} band data to assign `blue' and `red' suffixes.
Here, we build on the K10 scheme by proposing a method which is anchored in the optical and incorporates the spectral morphology in the \emph{H} and \emph{K} bands.
Below, we describe a classification method for estimating optically-anchored spectral types for L dwarfs based on NIR data using field and low-gravity templates and/or field- and low-gravity spectral standards.

\subsection{NIR spectral standards}
\subsubsection{Field-gravity standards}
\label{sec:field_stds}

In Figure \ref{fig:templates-stds} we compare the NIR field-gravity spectral average templates and the K10 spectral standards.
Except at L2 and L7, as described below, the agreement between the spectral morphology of the standards and the templates is excellent. The K10 standards were treated identically to the rest of L dwarfs when determining their inclusion in the optical-NIR sample used to calculate the NIR templates.
Six of the ten standards are included in the spectral average template calculation for their spectral type: \objectname[2MASS J03454316+2540233]{2M~0345+2540} (L0), \objectname[2MASS J21304464-0845205]{2M~2130-0845} (L1), \objectname[2MASS J15065441+1321060]{2M~1506+1321} (L3), \objectname[2MASS J21580457-1550098]{2M~2158-1550} (L4), \objectname[2MASS J10101480-0406499]{2M~1010-0406} (L6), and \objectname[2MASS J16322911+1904407]{2M~1632+1904} (L8).
The other four, however, are not included in the template for the K10 assigned NIR spectral type:
\begin{itemize}

	\item The L2 NIR standard, \objectname[2MASS J13054019-2541059]{Kelu-1}, a confirmed multiple system \citep{Liu05,Gelino06}, was rejected from the templates in \emph{H} and \emph{K} band, but not in \emph{zJ}. As can be seen in Figure~\ref{fig:templates-stds}, the spectrum of Kelu-1 is slightly different from the template: the blue slope of the \emph{H} band is steeper, the width of the \emph{H}-band peak is narrower, and the \emph{K}-band CO absorption is weaker than the template. While the spectrum is still the best fit to the L2 template at all bands and remains typed as an L2 in the NIR, it is not an ideal standard. Of the 13 L2s included in the template, we propose \objectname[2MASS J04082905-1450334]{2M~0408$-$1450} as an alternate L2 NIR spectral standard because it is bright, at moderate declination, and has a spectral morphology very similar to the L2 template (Figure~\ref{fig:templates-newstds}).

	\item The L5 NIR standard, \objectname[2MASS J08350622+1953050]{2M~0835+1953} is quite faint ($J=15.9$) and has no available optical spectrum. As a result, it is not included in the optical-NIR sample used as the basis for the template construction. As shown in Figure~\ref{fig:templates-stds}, despite not having a very high signal-to-noise ratio, the spectrum is a good match to the L5 template in all three bands. Since \objectname[2MASS J08350622+1953050]{2M~0835+1953} is so faint, we propose \objectname[2MASS J21373742+0808463]{2M~2137+0808} as a secondary L5 NIR standard. It is among the brightest objects included in the L5 template, is at moderate declination, and has a spectral morphology similar to the L5 template (Figure~\ref{fig:templates-newstds}).

	\item The L7 NIR standard, \objectname[2MASS J01033203+1935361]{2M~0103+1935}, is classified as L6 in the optical \cite{K00} and it is included in the L6 template.
	As can be seen in Figure~\ref{fig:templates-stds}, it is not a good match to the L7 template.
	Of the four objects included in the L7 template, we propose \objectname[2MASS J08251968+2115521]{2M~0825+2115} as an alternate L7 NIR spectral standard.
	It is the brightest of the four objects included in the template and has a NIR spectral morphology similar to them.
	Even though it is typed as an L7.5 in the optical, it remains the best candidate for an alternate L7 NIR spectral standard and we show it is a good match to the L7 template in Figure~\ref{fig:templates-newstds}.

	\item The L9 NIR standard, \objectname[2MASS J02550357-4700509]{2MASS 0255$-$4700}, is classified as an L8 in the optical and is included in the L8 template. (As stated earlier, there is no L9 in our optically anchored system.)

\end{itemize}

The updated list of NIR spectral standards reflecting our proposed revisions are listed in Table~\ref{tab:standards}.

\begin{figure}[hb!]
        \plotone{templates-stds.pdf}
        \caption{L dwarf \emph{zJ}, \emph{H}, and \emph{K}-band spectral templates (in black) with their flux range strips ---as described in \S~\ref{sec:templates}---overplotted with \replaced{their respective NIR spectral standard (in colors; \cite{Kirkpatrick10})}{the respective NIR spectral standard proposed by \cite{Kirkpatrick10} (in colors)}.
    \added{The K10 standards shown are 2M~0345+2540 (L0), 2M~2130-0845 (L1), Kelu-1 (L2), 2M~1506+1321 (L3), 2M~2158-1550 (L4), 2M~0835+1953 (L5), 2M~1010-0406 (L6), 2M~0103+1935 (L7), and 2M~1632+1904 (L8).}
    The width of the range (gray strips) indicates the range of spectral morphologies spanned
    by of the objects included in the template at each spectral type.
    Spectra are normalized over the entire band and offset to separate the spectra vertically.}
    \label{fig:templates-stds}
\end{figure}


\begin{figure}
        \plotone{templates-stdsL2L7.pdf}
        \caption{Proposed new NIR spectral standards compared to the spectral templates (black and gray).
        We propose 2M~0408-1450 (\replaced{green}{blue}), 2M~2137+0808 (orange), and 2M~0825+2115 (dark red) as L2, L5 and L7.5 NIR spectral standards respectively.\explain{Object names have been added.}}
    \label{fig:templates-newstds}
\end{figure}
\clearpage

\subsubsection{Low-Gravity standards}

We identify low-gravity spectral standards and list them in Table~\ref{tab:standards}.
We considered bright objects at intermediate declinations which were included in the spectral average templates.

Despite different selection criteria, four of our eight proposed low-gravity ($\gamma$ or $\beta$) standards are also proposed as \textsc{vl-g} or \textsc{int-g} standards by \citet{Allers:2013hk}, (hereafter AL13) and these are marked in Table~\ref{tab:standards} with a superscript `d'.
Despite having different classification methods\footnote{AL13 define a gravity score for each L dwarf spectrum by comparing several NIR spectral features against those of field dwarf spectra. The score of a spectral feature can be 0, 1, or 2, depending on how far its index or equivalent width measurement differs from the intrinsic scatter seen for field dwarfs. The median of all the spectral features' scores is then taken as the representative gravity score for the whole NIR spectrum. INT-G are objects having gravity scores consistent with intermediate gravity (median score of 1), and VL-G are objects having gravity scores consistent with very low gravity (median score $\geq$ 1.5).}, the AL13 \textsc{vl-g} and \textsc{int-g} suffixes are used to indicate essentially the same spectral morphology as our $\gamma$ and $\beta$ suffixes.
The L1$\gamma$, L2$\gamma$, and L3$\gamma$ standards we chose are the same as the \textsc{vl-g} L dwarfs proposed by AL13.
We chose different standards for L0$\gamma$ and L4$\gamma$.
\begin{itemize}

  \item For L0~\textsc{vl-g}, AL13 proposed \objectname[2MASS J22134491-2136079]{2M~2213$-$2136}. While this object's spectrum is representative of the type---it is included in the L0$\gamma$ template---we decided to choose two of the brighter L0$\gamma$ objects as standards. We chose the L0$\gamma$ prototype \objectname[2MASS J01415823-4633574]{2M~0141$-$4633} as an L0$\gamma$ standard and a second L0$\gamma$, \objectname[2MASS J17111353+2326333]{2M~1711+2326} with similar spectral morphology at a declination more easily accessible from the northern hemisphere.

  \item For L4~\textsc{vl-g}, AL13 proposed \objectname[2MASS J15515237+0941148]{2M~1551+0941} however, based on a new optical spectrum (\S~\ref{sec:obs_new_opt}), we type this object as L3.5$\gamma$ thus making it not an ideal L4$\gamma$ spectral standard.
  We propose \objectname[2MASS J05012406-0010452]{2M~0501-0010} as an L4$\gamma$ standard based on its spectral morphology, relatively bright \emph{J} mag, and intermediate declination.

\end{itemize}

AL13 identified five candidate \textsc{int-g} L dwarf spectral standards.
The L0$\beta$ standard we chose is the same as the L0~\textsc{int-g} they identified.
However, the four other objects they identify as possible \textsc{int-g} standards are problematic for various reasons.
\begin{itemize}

  \item They identified \objectname[2MASS J01174748-3403258]{2M~0117$-$3403} as a candidate L1~\textsc{int-g} standard; however, based on a new optical spectrum (\S~\ref{sec:obs_new_opt}), we type this objects as L1$\gamma$ and its NIR spectrum is included in the L1$\gamma$ template.

  \item They identified \objectname[2MASS J17260007+1538190]{2M~1726+1538} as a candidate L3~\textsc{int-g} standard; however, based on new optical data (\S~\ref{sec:obs_new_opt}), we type this object as L3.5$\gamma$.

  \item Two objects they identify as possible \textsc{int-g} standards, \objectname[2MASS J06023045+3910592]{2M~0602+3910} (L2) and \objectname[2MASS J01033203+1935361]{2M~0103+1935} (L6) do not have features of low gravity in the optical that would warrant a $\beta$ or $\gamma$ classification \citep{Cruz07,K00}.
\end{itemize}

In general (as noted by AL13) the objects classified as $\beta$ show more diversity in their spectral features compared to the $\gamma$ objects. Hopefully as more low-gravity L dwarfs are discovered and studied in detail, \added{the construction of} an intermediate gravity sequence will be possible.

\clearpage
%!TEX root = /Users/kelle/Dropbox/Pubs IN PROGRESS/Paper XII NIR DATA/Cruz-Nunez/NIRLdwarfs.tex
\begin{deluxetable}{lllllll}
%\tablewidth{0pt}
%\tabletypesize{\footnotesize}
%\tabletypesize{\small}
\tabletypesize{\scriptsize}
%\tabletypesize{\tiny}
%\rotate
\tablewidth{0pt}
\tablecolumns{6}
\tablecaption{Proposed Revised Set of L-Dwarf NIR Spectral Standards\label{tab:standards}}
\tablehead{ \multicolumn{3}{c}{2MASS All-Sky PSC} \\
\cline{1-3} & \colhead{}  & \colhead{} &  \colhead{Sp.} & \colhead{Proposed} & \colhead{Other}  \\
\colhead{Designation} & \colhead{$J$} &  \colhead{$J-K_S$}  &  \colhead{Type} & \colhead{Standard} & \colhead{Refs.}
}
\startdata
\multicolumn{6}{c}{Field Gravity} \\
\hline \\
03454316+2540233	& 13.997$\pm$0.027	& 1.325$\pm$0.036	& L0	& K10 & \cite{K99} \\
21304464$-$0845205	& 14.137$\pm$0.032	& 1.322$\pm$0.046	& L1	&  K10 & \cite{Kirkpatrick08,Reid08} \\
04082905$-$1450334\tablenotemark{a}	& 14.222$\pm$0.030	& 1.405$\pm$0.038	& L2	&   & \cite{Wilson01_thesis,Cruz03}\\
15065441+1321060	& 13.365$\pm$0.023	& 1.624$\pm$0.030	& L3	&  K10 & \cite{NN} \\
21580457$-$1550098	& 15.040$\pm$0.040	& 1.855$\pm$0.054	& L4	&  K10 & \cite{Cruz07, Kirkpatrick08} \\
08350622+1953050	& 16.094$\pm$0.075	& 1.775$\pm$0.090	& L5	&  K10 & \cite{Chiu06,Kirkpatrick10} \\
21373742+0808463\tablenotemark{b}	& 14.774$\pm$0.032	& 1.755$\pm$0.042	& L5	&   & \cite{Reid08} \\
10101480$-$0406499	& 15.508$\pm$0.059	& 1.889$\pm$0.075	& L6	&  K10 & \cite{Cruz03,Cruz07} \\
08251968+2115521\tablenotemark{c}	& 15.100$\pm$0.034	& 2.072$\pm$0.043	& L7.5	&   & \cite{K00} \\
16322911+1904407	& 15.867$\pm$0.070	& 1.864$\pm$0.085	& L8	&   K10 & \cite{K99} \\
\cutinhead{Low Gravity $\beta$} \\
15525906+2948485		 & 13.478$\pm$0.026	& 1.456$\pm$0.038	& L0$\beta$	&  AL13 & \cite{Reid08,Wilson03}      \\
02271036$-$1624479	& 13.573$\pm$0.023	& 1.430$\pm$0.038	& L1$\beta$	&   & \cite{Reid08,Deacon05} \\
\cutinhead{Low Gravity $\gamma$} \\
01415823$-$4633574			 & 14.832$\pm$0.043	& 1.735$\pm$0.054	& L0$\gamma$	&  & \cite{Cruz09_lowg, Kirkpatrick06} \\
17111353+2326333	& 14.499$\pm$0.026	& 1.443$\pm$0.037	& L0$\gamma$	&  & \cite{Cruz07}        \\
05184616$-$2756457	& 15.262$\pm$0.043	& 1.647$\pm$0.059	& L1$\gamma$	& AL13 & \cite{Cruz07}        \\
05361998$-$1920396	& 15.768$\pm$0.074	& 1.914$\pm$0.097	& L2$\gamma$	& AL13 & \cite{Cruz07}        \\
22081363+2921215	& 15.797$\pm$0.085	& 1.649$\pm$0.112	& L3$\gamma$	& AL13 & \cite{K00,Cruz09_lowg}  \\
05012406$-$0010452	& 14.982$\pm$0.038	& 2.019$\pm$0.052	& L4$\gamma$	&  & \cite{Reid08,Cruz09_lowg} \\
\enddata
\tablecomments{\explain{Table has been updated to reflect inclusion in K10 and AL13.}}
\tablenotetext{a}{Proposed replacement for Kelu-1 originally proposed as L2 NIR standard by K10.}
\tablenotetext{b}{Proposed secondary L5 standard.}
\tablenotetext{c}{Proposed replacement for 2M~0103+1935 originally proposed as L6 NIR standard by K10.}
\end{deluxetable}


\subsection{Method for estimating spectral types from NIR data}
\label{sec:typing}

In principle, we could set up three spectral typing schemes for the NIR, one for each band, but for simplicity, we propose one scheme that works for multiple or single band data.
We propose a spectral typing scheme that takes advantage of normalizing band-by-band (as described above in \S~\ref{sec:method}), uses spectral standards and the optically-anchored spectral average templates (\S~\ref{sec:templates}), and provides a method to assign spectral types to L~dwarfs with minimal NIR data:

\begin{itemize}
	\item If $JHK$ spectra are available, normalize the data at 1.28~$\micron$ and compare to the NIR spectral standards as described in K10. Note if the spectral slope is unusually red or blue compared to the standard.

	\item Normalize the spectra band-by-band as described in \S~\ref{sec:method} and compare to the spectral standards and/or the field- and low-gravity NIR templates to find the best matches in each band.
	The `best' match is the subtype where the spectral shapes look the most similar. The best match is usually bracketed by fair-to-good matches one type later and earlier than the best match.
	The fit in \emph{zJ}-band is given the most weight, followed by \emph{H} and \emph{K} bands.
	If multiple templates fit equally well in the \emph{zJ} band, then look for the best fit in the \emph{H} band.
	If multiple templates fit equally well in both \emph{zJ} and \emph{H}, then look for the best fit in \emph{K}.
	The goodness of the \emph{K} band match is considered last and relatively poor K-band matches are tolerated if there are good matches in the \emph{zJ} and \emph{H} bands.
	\begin{itemize}
		\item[$\circ$] The features in the \emph{zJ} band that determine a good match are the slope of the entire band and the strengths of the various absorption features; the 1.3~$\micron$ peak is not always matched well.
		\item[$\circ$] A good match in \emph{H} band is determined by matching the red and blue slopes on either side of the peak and the width of the peak; matching the steppe/shoulder or peak shape of the middle of the band (1.5--1.7~$\micron$) is not as crucial as matching the slopes.
		\item[$\circ$] The features contributing to a good match in the \emph{K} band are both the blue and red slopes on either side of the 2.13~$\micron$ peak.
	\end{itemize}
	\item If data are not available in all three bands (and \emph{zJ} band in particular) there will likely be a range of types that have equally good matches and the spectral type can be reported as a range (e.g., L3--L5). In these cases, the range notation is preferred since L3--L5 implies the types L3, L4, and L5 are all equally probable, whereas L4$\pm$1 implies a best-fit type of L4 using a method with one spectral type precision.
	\item If the target is not a good match to any available standards or field or low-gravity templates, consider a `pec' suffix to indicate the peculiarity of the spectrum. If the spectrum is also unusually red or blue when normalized at 1.28~$\micron$, consider a `red' or `blue' suffix.
	\item Spectral types based on low signal-to-noise data ($S/N\sim15$) can be indicated with a `:' suffix or `::' for very-low signal-to-noise data ($S/N\sim5$).
    For example, L1: could be used to describe a low signal-to-noise spectrum that matches the L1 template best in both \emph{zJ} and \emph{H} bands. L1--L3:: could be used to describe a very low signal-to-noise \emph{K}-band spectrum that matches the L1, L2, and L3 templates all equally well.
	The use of colons to indicate poor quality data is common practice in stellar astronomy, e.g., \citet{Sanduleak:1988fn}, \cite{Silvestri:2006el}, \citet[Table 12.1]{Gray:2009wd}, and \cite{Covey:2010de}.

\end{itemize}

Presently, we do not encourage the use of half-types (or any further gradation) for spectral types based on low-resolution NIR data.
Compared to the optical sequence, where half-types are warranted, the low-resolution NIR data display a greater diversity of spectral morphologies at each type and the resulting sequence is coarser.
In addition, the three component piece-wise normalization method is not fully accounting for the color term of each object and the residual reddening is likely contributing to the spread of spectral morphologies at each spectral type.

We have developed a Python program, dubbed The Ultracool Typing Kit \citep{Schwab:2017ex}, which facilitates the spectral comparisons described above.
The program normalizes the input NIR spectrum band-by-band (as described in \S~\ref{sec:method}) and overplots it on the grid of the spectral field- and low-gravity templates listed in Table~\ref{tab:standards}.
The program, a more detailed description, and usage instructions can all be found on GitHub\footnote{The Ultracool Typing Kit: \url{https://github.com/BDNYC/UltracoolTypingKit}}.

\subsubsection{New method compared to optical and literature NIR types}

We use the optical-NIR sample described in \S~\ref{sec:sample} to compare the spectral typing method described above to optical spectral types and NIR spectral types found in the literature.
The NIR spectra in the optical-NIR sample were stripped of identifying information and typed using the method described above (via the Ultracool Typing Kit).

As described in  \S~\ref{sec:sample}, we used SPLAT to identify NIR spectral types existing in the literature for 78~objects in the optical-NIR sample.
We compare our NIR spectral types to those literature NIR types in Figure~\ref{fig:Lit NIR comparison}.
While the agreement is generally good, with 58/78 (74\%) agreeing within one subtype, we note a strong tendency for the literature types to be later than the ones found with the method proposed here.
This is not surprising since this same trend is observed with the optical types (Figure~\ref{fig:opt_lit_nir_comparison}) to which our scheme is anchored.

The resulting NIR types are compared to the optical types in Figure~\ref{fig:optical NIR comparison}.
In general, the agreement is excellent, with 90\% of NIR types being within one subclass of the optical type and having the same suffix (153/\sample).
Only 6/\sample~(4\%) objects are off by more than one subtype and 13/\sample~(8\%) were assigned a different suffix.
This analysis shows that the NIR spectral typing method we propose here does indeed result in NIR types which are consistent with the optical types.

\begin{figure}
  \plotone{nir_field_obj.pdf}
  \caption{Swarm plot representation of literature NIR spectral types compared to NIR types found using the method described here.
  Filled orange, blue, and pink circles indicate objects where the suffixes agree while a black star indicates suffix disagreement.
  While the agreement is generally good, with 58/78~(74\%) agreeing within one subtype, we note a strong tendency for the literature types to be later than the ones found with the method proposed here.}
  \label{fig:Lit NIR comparison}
\end{figure}

\begin{figure}
  \plotone{opt_field_obj.pdf}
  \caption{Swarm plot representation of optical spectral types compared to NIR types found using the method described here.
  Filled orange, blue, and pink circles indicate objects where the suffixes agree while a black star indicates suffix disagreement.
  In general, the agreement is excellent, with 90\% of NIR types being within one subclass of the optical type and having the same suffix (153/\sample).
  Only 6/\sample~(4\%) objects are off by more than one subtype and 13/\sample~(8\%) were assigned a different suffix.}
  \label{fig:optical NIR comparison}
\end{figure}

\section{Summary, Conclusions, and Future Work}
\label{sec:summary}
We present analyses of \sample~L dwarfs with both high quality optical and NIR spectra.
The sample is comprised of \optField~field-gravity L0--L5 objects and \optLowG~L0--L4 low-gravity $\beta$~and $\gamma$~objects.
Included in the sample are new low-resolution NIR SpeX spectra for \NewPrismObjects~L dwarfs and new moderate-resolution far-red optical spectra for~\NewOptObjects.
\added{These data are all are available as Data behind the Figures and from the BDNYC Database.}
In our analysis of the NIR spectra, we normalize the spectra band-by-band rather than at one small wavelength window.
This normalization method effectively reduces the broad-band reddening term present in the spectra and facilitates the comparison of the molecular absorption features.

We have analyzed the NIR spectra of L~dwarfs using optical data as an anchor.
We used an iterative method to combine the NIR spectra into \emph{zJ}, \emph{H}, and \emph{K} band spectral average templates at each spectral type: L0--L8, L0--L4$\gamma$, and L0--L1$\beta$.
While there is significant homogeneity among the spectral morphologies at each spectral type (by definition), there is also a spread in these morphologies within each spectral type.
This range is likely dominated by intrinsic properties, such as clouds, gravity, and metallicity, and not instrumental or (terrestrial) atmospheric effects.
We compared the low-gravity templates with the field templates and developed recommendations for credibly distinguishing low-gravity objects from field objects using NIR spectra.

Our analysis reveals several interesting things about L dwarf properties:
\begin{itemize}
	\item L dwarfs of the same optical spectral type do indeed have common spectral morphologies in the NIR even though they span a wide range of colors. The $J-K$ color scatter amongst L dwarfs of the same optical spectral type is attributable to the spectra having different smooth functions across the entire 1--2.5~$\micron$ range rather than differences amongst individual absorption features.
	\item There are objects with quite red colors for their spectral type that do not have any recognized signatures of low gravity in their optical or low-resolution NIR spectra. This further supports the notion that most low-gravity objects are red, but not all red objects are low gravity.
	\item Some objects excluded from the templates have near-average colors. This implies that not all objects with peculiar spectra have extreme colors.
	\item Some objects with colors in the red/blue tails of the color distribution are included in the templates. This implies that not all objects with extreme colors display significant spectral peculiarities in low-resolution NIR data besides spectral slope.
	\item There is more scatter in the $J-K$ colors of the later-type field-gravity objects (L4--L7) compared to the earlier types (L0--L3). This could be attributed to the larger range of underlying masses, ages, and gravities expected in the brown dwarf population and/or to the increased sensitivity of the emergent spectra to cloud properties at cooler temperatures.
	\item The optically anchored NIR low-gravity $\gamma$ low resolution spectral sequence appears to be coarser than the field-gravity sequence.
\end{itemize}

We have used the optically-anchored NIR spectral average templates to evaluate the current suite of field and low-gravity NIR spectral standards proposed by K10 and AL13.
We have suggested alternate and secondary standards where previously proposed standards have turned out to be non-representative of the population and/or where brighter, more accessible suitable standards exist.

We proposed a scheme to assign spectral types to low-resolution NIR spectra of L~dwarfs.
The scheme
\begin{itemize}
  \item results in spectral types consistent with optical types,
  \item is anchored to the optical and thus primarily correlated with temperature,
  \item provides a method to estimate spectral types using spectra with limited wavelength coverage,
  \item requires an analysis of the spectra band-by-band and thus is minimally sensitive to the range of NIR colors spanned by each type,
  \item utilizes high-fidelity spectral average templates in addition to spectral standards, and
  \item encourages the use of spectral ranges and colon suffixes to clearly convey the accuracy of the assigned types.
\end{itemize}

Future work will include expanding the optical-NIR sample low-resolution SpeX spectra to include more objects, especially low-gravity objects and later-type field-gravity objects.
Based on the homogeneity present among the spectral morphologies of the current sample, we do not expect the templates to change significantly as more objects are added.
However, more objects are needed to further investigate the trends that we have observed in $J-K$ color, especially at the later spectral types.
Further, more objects and analysis are required to make the low-gravity $\beta$~spectral sequence more robust and to extend the $\gamma$~sequence later than L4$\gamma$.

Our analysis suggests that a great deal of further theoretical and observational work is needed to pinpoint the nature of the physical properties modulating the spectral morphologies and colors of L~dwarfs.
However, our results provide methods to reliably and consistently identify and classify L~dwarf spectra and represent significant progress in our ability to decode NIR spectra to reveal the underlying physics of L~dwarf atmospheres.

\clearpage
\appendix
\section{NIR Spectra of Galaxies}
\label{sec:galaxies}
As part of the spectroscopic follow-up described in \S~\ref{sec:obs_new_nir}, some candidates turned out to be likely galaxies and not M or L~dwarfs. We list these objects in Table~\ref{tab:notm} and show their spectra in Figure Set~\ref{fig:notMs_1}.
\added{These data are available as Data behind the Figure.}

%!TEX root = /Users/kelle/Dropbox/Pubs IN PROGRESS/Paper XII NIR DATA/Cruz-Nunez/NIRLdwarfs.tex
\begin{deluxetable}{ll}
\tablecolumns{2}
\tablecaption{Near-Infrared IRTF SpeX Prism Spectra of Galaxies\label{tab:notm}}
\tablehead{\colhead{2MASS} & \colhead{Obs. Date} \\
\colhead{Designation} & \colhead{(UT)}
}
\startdata
00110940+5149236 & 2003 Sep 05 \\
02561474+1935213 & 2003 Sep 05 \\
02562362+2013457 & 2003 Sep 05 \\
04214954+1528598 & 2004 Nov 07 \\
05151593$-$0656262 & 2004 Nov 07 \\
05574102$-$1333264 & 2004 Nov 07 \\
05574229$-$1333156 & 2003 Sep 05 \\
05583706$-$1339123 & 2003 Sep 05 \\
12475047$-$0152142 & 2005 Mar 23 \\
12490476$-$3454014 & 2005 Mar 23 \\
13570485$-$3946356 & 2005 Mar 23 \\
15063706+2759544 & 2003 Sep 05 \\
17213581$-$0619145 & 2003 Sep 03 \\
20151370$-$1252571 & 2003 Aug 11 \\
22400942+3848306 & 2003 Aug 12 \\
22433237$-$1525260 & 2003 Aug 12 \\
23490528+1833150 & 2003 Sep 04 \\
\enddata
\tablecomments{Spectra shown in Figure Set~\ref{fig:notMs_1}.}
\end{deluxetable}


\figsetstart
\figsetnum{A1}
\figsettitle{Normalized NIR spectra of 2MASS targets identified as likely galaxies.}

\figsetgrpstart
\figsetgrpnum{A1.1}
\figsetgrptitle{First set of likely galaxies}
\figsetplot{notM_galaxies1.pdf}
\figsetgrpnote{Normalized NIR spectra of 2MASS targets identified as likely galaxies.}
\figsetgrpend

\figsetgrpstart
\figsetgrpnum{A1.2}
\figsetgrptitle{Second set of likely galaxies}
\figsetplot{notM_galaxies2.pdf}
\figsetgrpnote{Normalized NIR spectra of 2MASS targets identified as likely galaxies.}
\figsetgrpend

\figsetgrpstart
\figsetgrpnum{A1.3}
\figsetgrptitle{Third set of likely galaxies}
\figsetplot{notM_emln_galaxies.pdf}
\figsetgrpnote{Normalized NIR spectra of 2MASS targets identified as likely galaxies.}
\figsetgrpend

\figsetend

\begin{figure}
    \epsscale{1}
    \plotone{notM_galaxies1.pdf}
    \caption{Normalized NIR spectra of 2MASS targets identified as likely galaxies. \added{These data are available as Data behind the Figure.} Additional images are available in the Figure Set in the electronic edition of the journal.}
    \label{fig:notMs_1}
\end{figure}

\begin{figure}
    \figurenum{A1.2}
    \plotone{notM_galaxies2.pdf}
    \caption{Normalized NIR spectra of 2MASS targets identified as likely galaxies cont. Will be electronic only.}
\end{figure}

\begin{figure}
    \figurenum{A1.3}
    \plotone{notM_emln_galaxies.pdf}
    \caption{Normalized NIR spectra of 2MASS targets identified as likely galaxies cont. Will be electronic only.}
\end{figure}
\clearpage

\acknowledgments

The authors would like to thank J. Davy Kirkpatrick, Michael Cushing, \added{and Jonathan Gagn\'{e} for useful discussions and Eileen Gonzales and Victoria DiTomasso for assistance compiling the references}.
This material is based upon work supported by the National Science Foundation under Grant No. 1313132.
This research has benefited from the SpeX Prism Spectral Libraries, maintained by Adam Burgasser
and made use of NASA's Astrophysics Data System; the SIMBAD database, operated at CDS, Strasbourg, France; and the AstroBetter blog and wiki.
Some of the data presented herein were obtained at the W.~M.~Keck Observatory, which is operated as a scientific partnership among the California Institute of Technology, the University of California and the National Aeronautics and Space Administration. The Observatory was made possible by the generous financial support of the W.~M.~Keck Foundation.
This paper also includes data gathered with the 6.5 meter Magellan Telescopes located at Las Campanas Observatory, Chile.
This publication makes use of data products from the Two Micron All Sky Survey, which is a joint project of the University of Massachusetts and the Infrared Processing and Analysis Center/California Institute of Technology, funded by the National Aeronautics and Space Administration and the National Science Foundation.
The authors wish to recognize and acknowledge the very significant cultural role and reverence that the summit of Mauna Kea has always had within the indigenous Hawaiian community. We are most fortunate to have the opportunity to conduct observations from this mountain.

\facility{IRTF(SpeX), Keck(LRIS), Magellan:Clay(LDSS3)}

\added{\software{Astropy \citep{TheAstropyCollaboration:2013cd},
          Matplotlib \citep{matplotlib:2007ih}, NumPy and SciPy \citep{numpy:2011}, Seaborn \citep{Seaborn:2017gc}}
          }

\bibliographystyle{aasjournal}
\bibliography{paper12}
%export from Papers. Paper12 Refs Collection, Standard, Abbreviation

\listofchanges

\end{document}
