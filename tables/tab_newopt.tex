%!TEX root = /Users/kelle/Dropbox/Pubs IN PROGRESS/Paper XII NIR DATA/Cruz-Nunez/NIRLdwarfs_Normal.tex
\begin{deluxetable}{lcclcllll}
%\tablewidth{0pt}
%\tabletypesize{\footnotesize}
%\tabletypesize{\small}
\tabletypesize{\scriptsize}
%\tabletypesize{\tiny}
%\rotate
\tablewidth{0pt}
\tablecolumns{9}
\tablecaption{New Far-Red Optical Observations\label{tab:newopt}}

\tablehead{ 

\colhead{} & \colhead{} &  \colhead{} & \colhead{Obs Date}  & \colhead{Optical} & \colhead{Sp. Type} & \colhead{Other} \\

\colhead{Object} & \colhead{Telescope} & \colhead{Instrument} &  \colhead{(UT)} & \colhead{Sp. Type\tablenotemark{b}} & \colhead{Ref.} & \colhead{Refs.\tablenotemark{c}} 
}
\startdata     
00325584-4405058  &   Keck I	& LRIS &	2009 Oct 11 & L0$\gamma$	& \cite{Cruz09_lowg} &	\cite{Reid08}             \\
00550564+0134365  &   Keck I	& LRIS &	2009 Oct 11 & L2$\gamma$	& This paper &	              \\
0117474-340325    &   Keck I	& LRIS &	2009 Oct 12 & L1$\beta$	& \cite{Cruz03} &	\cite{Cruz03}	            \\
%0117474-340325    &   Keck I	& LRIS &	2009 Oct 12 & L1$\beta$	& This paper &	Cruz03              \\
02103857-3015313  &   Keck I	& LRIS &	2009 Oct 11 & L0$\gamma$	& This paper &	             \\
0241115-032658    &   Keck I	& LRIS &	2009 Oct 12 & L0$\gamma$	& \cite{Cruz09_lowg} &	\cite{Cruz07}              \\
%0436278-411446    &  Keck I	& LRIS &	2009 Oct 11 & M8β	& \cite{Cruz07} &	NN5                 \\
%0436278-411446    &  Keck I	& LRIS &	2009 Oct 11 & M8β	& This paper &	NN5                 \\
%0512063-294954    &   GS		& GMOS &	2005 Oct 10 & L5$\gamma$	& Kirkpatrick08	& Cruz03        \\
0518461-275645    &   Keck I	& LRIS &	2009 Feb 17  & L1$\gamma$	& This paper	& \cite{Cruz07}            \\
%05341594-0631397  &  Keck I	& LRIS &	2009 Oct 11 & M8γ	& Kirkpatrick10 &	Kirkpatrick1\\  
0536199-192039    &   Keck I	& LRIS &	2009 Feb 17  & L2$\gamma$	& This paper	& \cite{Cruz07}            \\
13313310+3407583  &   Keck I	& LRIS &	2009 Feb 17  & L0	& \cite{Reid08}	& \cite{Reid08}           \\
15382417-1953116  &   Keck I	& LRIS &	2010 Jul 17  & L4$\gamma$	& This paper	&            \\
15515237+0941148  &   Keck I	& LRIS &	2010 Jun 18  & L3.5$\gamma$	& This paper	& \cite{Reid08}           \\
\nodata  &   Keck I	& LRIS &	2009 Feb 17  & L3.5$\gamma$	& This paper	& \cite{Reid08}           \\
1615425+495321    &   Keck I	& LRIS &	2010 Jun 18  & L4$\gamma$	& This paper	& \cite{Cruz07}            \\
\nodata    &   Keck I	& LRIS &	2009 Feb 17  & L4$\gamma$	& This paper	& \cite{Cruz07}            \\
1711135+232633    &   Keck I	& LRIS &	2010 Jun 18  & L0$\gamma$	& This paper	& \cite{Cruz07}            \\
1726000+153819    &   Keck I	& LRIS &	2010 Jun 18  & L3.5$\gamma$ &	\cite{Cruz09_lowg}	& \cite{K00}               \\
%19355595-2846343  &  Keck I	& LRIS &	2009 10 11 & M9γ	& This paper	& \cite{Reid08}           \\
%2002507-052152    &   Keck I	& LRIS &	2010 Jun 18  & L5$\beta$	& This paper	& \cite{Cruz07}            \\
20343769+0827009  &   Clay 		& LDSS-3 &  2008 Oct 07  & L1	& This paper	&             \\
21373742+0808463   & Keck I	& LRIS & 2010 Jul 17 & L5 & This paper & \cite{Reid08} \\	
2154345-105530    &   Keck I	& LRIS &	2009 Oct 11 & L4$\beta$	& This paper &	              \\
23153135+0617146  &  Keck I 	& LRIS &	2009 Oct 11 & L0$\gamma$ &	This paper &		        \\
\enddata

\tablecomments{Spectra shown in Figure~\ref{fig:newopt}.}

% \tablenotetext{a}{The sexagesimal right ascension and declination suffix of the full 2MASS All-Sky Data Release designation 
% (2MASS Jhhmmss[.]$\pm$ssddmmss[.]s) is listed for each object.
% The coordinates are given for the J2000.0 equinox; the units of right ascension are hours, minutes, and seconds; 
% and units of declination are degrees, arcminutes, and arcseconds.}
% 
% \tablenotetext{c}{Uncertainties on spectral types are $\pm$ 0.5 subtypes except where noted by one or two colons, indicating an uncertainty of $\pm$1 and $\pm$2 types respectively.}
% 
 \tablenotetext{c}{References listed pertain to discovery and previous \emph{optical} classifications.} 
% %See \citet{Reid06_binary} for objects targeted with NICMOS but unresolved.}
% 
% \tablenotetext{d}{Distance estimate based on trigonometric parallax.}
% 

\end{deluxetable}
